\documentclass{svmult}

\usepackage{mathptmx}       % selects Times Roman as basic font
\usepackage{helvet}         % selects Helvetica as sans-serif font
\usepackage{courier}        % selects Courier as typewriter font
\usepackage{type1cm}        % activate if the above 3 fonts are
                            % not available on your system
\usepackage[numbers]{natbib}
%
\usepackage{makeidx}         % allows index generation
\usepackage{graphicx}        % standard LaTeX graphics tool
                             % when including figure files
\usepackage{multicol}        % used for the two-column index
\usepackage[bottom]{footmisc}% places footnotes at page bottom
\usepackage{listings}
\usepackage{color}
\usepackage{tikz}
\usepackage{amsmath}

\usepackage{footmisc}
\usepackage{import}
\usepackage{enumitem}
\usepackage{booktabs}

\usepackage{capt-of}
\usepackage{float}
\usepackage{tabularx}
\usepackage{watermark} % titlepage image
\usepackage{booktabs}
\usepackage{hyperref}
\usepackage[parfill]{parskip}

\usetikzlibrary{shapes.misc, arrows, positioning, calc, decorations.markings}
\tikzset{
  query/.style={draw=blue!10,thick,fill=blue!2,inner sep=.15cm},
  answer/.style={rectangle,draw=black!10,fill=gray!4},
  icon/.style={circle,thick,fill=blue!20,draw=blue!30,inner sep=.05cm,font=\bfseries}
}
\pgfdeclarelayer{back}
\pgfdeclarelayer{front}
\pgfsetlayers{back,main,front}

\lstdefinelanguage[OWL]{XML} {morekeywords={Individual,ObjectProperty,Types,Facts,Class,SubClassOf,Domain,Range,SubPropertyOf,EquivalentTo}}
\lstdefinestyle{OWL}    {language=[OWL]XML,    lineskip=0.2ex, fontadjust=true, basicstyle={\scriptsize \nopagebreak[4]}}
\lstdefinestyle{Prolog} {language=Prolog, lineskip=0.2ex, fontadjust=true, basicstyle={\scriptsize \nopagebreak[4]}, commentstyle=\scriptsize,
    morekeywords={entity, occurs, holds, show, append, forall, findall, member}}

\graphicspath{img}

\lstdefinestyle{lispcode}{
	backgroundcolor=\color{lightgray},   
	commentstyle=\color{codegreen},
	keywordstyle=\color{magenta},
	numberstyle=\tiny\color{white},
	stringstyle=\color{purple},
	basicstyle=\ttfamily\footnotesize,
	breakatwhitespace=false,         
	breaklines=true,                 
	captionpos=b,                    
	keepspaces=true,                 
	numbers=left,                    
	numbersep=5pt,                  
	showspaces=false,                
	showstringspaces=false,
	showtabs=false,                  
	tabsize=2
}

\newcommand{\todo}[1]{\textcolor{red}{\textbf{TODO}: #1}}
%\lstdefinelanguage[OWL]{XML} {morekeywords={Individual,ObjectProperty,Types,Facts,Class,SubClassOf,Domain,Range,SubPropertyOf,EquivalentTo}}
%\lstdefinestyle{OWL}    {language=[OWL]XML,    lineskip=0.2ex, fontadjust=true, basicstyle={\scriptsize \nopagebreak[4]}}
%\lstdefinestyle{Prolog} {language=Prolog, lineskip=0.2ex, fontadjust=true, basicstyle={\scriptsize \nopagebreak[4]}, commentstyle=\scriptsize,
%    morekeywords={entity, occurs, holds, show, append, forall, findall, member}}

\title*{NEEM Specification}
\author{The EASE Researchers}
\institute{CRC Everyday Activity Science and Engineering (EASE)\\ University Bremen, Am Fallturm 1, 28359 Bremen\\ \texttt{ai-office@cs.uni-bremen.de}}
\thiswatermark{
 \centering
 \put(-50,-500){\includegraphics[width=14cm]{img/NarrativesStory.png}}
 \put(340,-10){\includegraphics[width=4.0cm]{img/ease-logo.pdf}}}

\newcommand{\neemversion}{0.1~}
\newcommand{\neemexp}{NEEM-experience~}
\newcommand{\neemexps}{NEEM-experiences~}
\newcommand{\neemnar}{NEEM-narrative~}
\newcommand{\neem}{NEEM~}
\newcommand{\neems}{NEEMs~}
\newcommand{\neemhub}{NEEM-hub~}
\newcommand{\tf}{tf~}
\newcommand{\openease}{openEASE~}
\newcommand{\ease}{EASE~}
\newcommand{\mongodb}{MongoDB~}
\newcommand{\cram}{CRAM~}
\newcommand{\knowrob}{KnowRob~}
\newcommand{\owl}{OWL~}
\newcommand{\pr}{PR2~}
\newcommand{\qudt}{Qudt~}
\newcommand{\boxy}{Boxy~}
\newcommand{\eg}{e.g.~}
\newcommand{\ros}{ROS~}
\newcommand{\soma}{SOMA~}
\newcommand{\easeOwl}{EASE.owl~}
\newcommand{\easeAct}{EASE-ACT.owl~}
\newcommand{\easeObj}{EASE-OBJ.owl~}
\newcommand{\cramOwl}{cram\_failures.owl}
\newcommand{\owlClass}[1]{\textit{#1}}
\newcommand{\owlPredicate}[1]{\textit{#1}}

\renewcommand\tabularxcolumn[1]{m{#1}}
\newcolumntype{Y}{>{\centering\arraybackslash}X}


\begin{document}
\maketitle

\chapter{Introduction}
\chapterauthor{D. Be{\ss}ler, S. Koralewski, M. Pomarlan}

This document, referred to as the ``\neem Handbook'' hereafter,
describes the \ease system for episodic memories of everyday activities.
%The \neem Handbook will be updated along the progress in the CRC \ease.
It is thought to provide \ease researchers with compact but still comprehensive
information about what information is contained in \neems,
how it is represented, acquired, curated and published.

\begin{figure}[h!]
\centering
\begin{tikzpicture}[
    border/.style={draw=easeblue},
    dark/.style={border,fill=easeblue!20},
    light/.style={border,fill=easeblue!10},
    lighter/.style={border,fill=easeblue!5},
    heading/.style={text=easeblue,font=\bf,text badly centered},
    label1/.style={text=easeblue,text badly centered},
    label2/.style={text=black,text badly centered,font=\it},
    label/.style={text=black,text width=1.7cm,text badly centered},
    hexagon/.style={regular polygon,regular polygon sides=6,rounded corners},
    triangle/.style={regular polygon,regular polygon sides=3,rounded corners},
    trapez/.style={font=\bf,text badly centered,trapezium,trapezium angle=60,rounded corners},
    user/.style={hexagon,lighter,minimum width=2.5cm,inner sep=0},
    img/.style={}
]
  %\draw[help lines](-4,-4)grid(4,3);
  %% outer hexagon
  \node[hexagon,dark,minimum width=5.0cm] (OUTER) {};
  \node[heading] at (0,-1.8) (ACQ) {acquisition};
  \node[heading,rotate around={120:(OUTER)},rotate=180] at (ACQ) {curation};
  \node[heading,rotate around={-120:(OUTER)},rotate=180] at (ACQ) {publication};
  %% inner triangle
  \node[triangle,light,minimum width=6.0cm] at (OUTER) (INNER) {};
  %% inner trapez
  \node[trapez,lighter,
    minimum height=21mm,yshift=-0.275cm] at (INNER.center) (LIB) {};
  \node[label,yshift=0.2cm] at (INNER) (BG) {NEEM Background};
  \node[label,below=0.2cm of BG.south,anchor=north east] (NAR) {NEEM Narrative};
  \node[label,below=0.2cm of BG.south,anchor=north west] (EXP) {NEEM Experience};
  \draw[draw=easeblue] (BG.south west) -- (BG.south east);
  \draw[draw=easeblue] (NAR.north east) -- (NAR.south east);
  %% heading
  \node[label1,text width=14mm,above=0.4cm of BG] (HUB1) {NEEM HUB};
  %%
  \node[user,text=black,anchor=north,yshift=-0.2cm,
    path picture={\node[yshift=1.8cm] at (path picture bounding box.center){
        \includegraphics[width=2.55cm]{img/topf-2.jpg}
    };}
  ] at (OUTER.south) (X1) {};
  \node[user,anchor=center,yshift=-0.4cm,xshift=2.4cm,
    path picture={\node[yshift=0.0cm] at (path picture bounding box.center){
        \includegraphics[height=2.55cm]{img/pr2_milk_frosties_small_top.jpg}
    };}
  ] at (X1.north) (X2) {};
  \node[user,anchor=center,yshift=-0.4cm,xshift=-2.4cm,
    path picture={\node[yshift=0.0cm] at (path picture bounding box.center){
        \includegraphics[height=2.55cm]{img/cup-on-plate-unstable-final.jpg}
    };}
  ] at (X1.north) (X5) {};
  \node[user,rotate around={120:(OUTER)},
    path picture={\node[yshift=0.0cm] at (path picture bounding box.center){
        \includegraphics[height=2.55cm]{img/Programmer.png}
    };}
  ] at (X1) (X3) {};
  \node[user,rotate around={-120:(OUTER)},
    path picture={\node[yshift=0.0cm] at (path picture bounding box.center){
        \includegraphics[height=2.55cm]{img/open-ease-web-page-full_cropped.png}
    };}
  ] at (X1) (X4) {};
  %%
  \node[label2,anchor=north] at (X1.south) {observation};
  \node[label2,anchor=north] at (X2.south) {experimentation};
  \node[label2,anchor=north] at (X5.south) {simulation};
  
\end{tikzpicture}
\caption{The EASE system for acquisition, curation and publication of episodic memories}
\label{fig:architecture}
\end{figure}

\paragraph{Narrative Enabled Episodic Memories}
% From KnowRob 2.0
When somebody talks about the deciding goal in the last soccer world championship many of us can ``replay'' the episode in our ``mind's eye''.
Those episodic memories can be seen as abstract descriptions that allow us to recall detailed pieces of information from any experienced activity.
Having those detailed memories, we can use them to learn general knowledge or map similar memories to unknown situations, so we know how to behave in the given situation.
\todo{DB: refer to Figure~\ref{fig:architecture}}

% From KnowRob 2.0
\ease integrates episodic memories deeply into the knowledge acquisition, representation, and processing system. 
For every activity the agent performs, observes, prospects and reads about, it creates an episode and stores it in its memory.
An episode is best understood as a video recording that the agent makes of the ongoing activity.
In addition, those videos are enriched with a very detailed story about the actions, motions, their purposes, effects and the agent's sensor information during the activity.

% From KnowRob 2.0
We define the episodic memories created by our system narrative-enabled episodic memories (\neems).
A \neem consists of the \emph{\neem experience} and the \emph{\neem narrative}.
The \neem experience captures low-level data such as the agent's sensor information, e.g. images and forces, and records of poses of the agent and its detected objects.
\neem experiences are linked to \neem narratives, which are stories of the episode described symbolically.
These narratives contain information regarding the tasks, the context, intended goals, observed effects, etc.
The \neemexp and \neemnar combined are so rich of information that the agent can replay an episode to experience the seen activity anytime again.

\neems are representations of experiences acquired through experimentation, reading, observing, mental simulation, etc.
The main goal is to establish a common vocabulary used to annotate experience data across different tasks, scientific disciplines, and modalities of acquisition, and to define models for the representation of experience data.
The vocabulary is not just a set of atomic labels, but each label has a formal definition in an ontology.
These definitions are done such that a set of \emph{competency questions} about an activitiy can be answered by a knowledge base that is equipped with the ontology and a collection of \neems.

The \neem model is formally defined in form of an \owl ontology which is based on the DOLCE+DnS Ultralite (DUL) upper-level ontology~\cite{DOLCE2003}.
DUL is a carefully designed ontology that seeks to model general categories underlying human cognition without making any discipline-specific assumptions.
Our extensions of DUL mainly focus on characterizing different aspects of activities that were not considered in much detail in DUL, but are relevant for the autonomous robotics scope.
These extensions are part of an ontology that we have called
\soma~\footnote{\url{https://ease-crc.github.io/soma}}.
A \neem is made of several patterns defined either in \dul or in \soma.

While it is possible to create the representations listed in this document through a custom exporter, it is not advised to do so.
Instead, it is advised to interface with the
\knowrob knowledge base~\footnote{\url{https://github.com/knowrob/knowrob}}.
\knowrob provides an interface based on predicate logics that allows to interact with \neems.
The language is a collection of predicates that can be called by users to ask certain types of competency questions covering different aspects of activitiy, or to add labels and relationships in the \neemnar.
We will provide example expressions in this document that highlight how the knowledge base can be used to interact with \neems.

% paragraph about generalizability of robot behaviour (add reference to Pratt paper ``Is a Cambrian Explosion Coming for Robotics?'')
\lipsum[4]
\todo{DB: write paragraph about generalizability, Pratt paper ``Is a Cambrian Explosion Coming for Robotics?''}

% % % % % % % % % % % % % % % % %
% % % % % % % % % % % % % % % % %
\section{Notation} % Domain of Discourse
\label{sec:notation}

In this Section, we will shortly introduce the notions and notations that are important to follow this document.
%Because in subsequent sections we will give more formal definitions for various concepts, we now review some basic notions about the main formalism we use in the ontology, as well as introduce the notation we will employ.

%%% ABOX -- TBOX
\neems are formally represented using an \emph{ontology}.
An ontology is a collection of logical axioms in some formal language such as description logic (DL).
%In the case of many ontologies available in the semantic web, as well as in the case of SOMA, this formal language is description logic, also known as DL.
The entities that can be described in DL can be either \emph{concepts} (sometimes known as \emph{classes}),
and \todo{Seba: Can we here remove individuals?
It seems not very well definied that concepts are known as individuals and in the next sentence the instances are called individuals.} \emph{individuals}.
An individual is an instance of one or more concepts.
A concept may be subsumed by another.
Between individuals there may be relations called \emph{object properties},
and, in addition, an individual can also have \emph{data properties} that link it to some data values.
%As a syntactic convenience, an individual can also have ``data properties'' that link it to some alphanumeric data item which is itself not considered an individual in the ontology.
As an example, let us assume that Alice and Bob are both individuals of the concept \concept{Human},
and that the object property \relation{hasChild} connects Alice to Bob,
i.e. the relation asserts that Bob is a child of Alice.
We may also know the height of Alice, which would be represented by a data property \relation{hasHeight} whose value could be a string such as \emph{1,7m} to represent that she is 1.7m tall.
%%
In the following, to make clear when we are talking about concepts and when about individuals, we will denote the set of all concepts as $\mathcal{T}$ (called the TBox), and the set of all individuals as $\mathcal{A}$ (called the ABox).
%Accordingly, there are axioms that describe concepts or object/data properties,
%and axioms that describe individuals and the relations between them.
%The set of concepts, object and data properties, and the axioms that
%define them is called the Tbox, or terminological part,
%and we will denote it by $\mathcal{T}$.
%The set of individuals and the axioms describing them is called the Abox,
%or assertion part, and we will denote it by $\mathcal{A}$.

%%% formatting
It is useful when describing concepts to emphasize the concept names such that it is clear we reference the concept, and not the colloquial word. As such, \concept{Concepts} and \relation{relations} will be written in a different font.
Note that the name of a concept always starts with an uppercase letter, whereas the name of a relation with a lowercase one.
Any word appearing in a concept or relation name after the first one will always begin with an uppercase letter.

%%% namespaces
Ontologies are meant to build on one another, and it is not uncommon for an ontology to collect thousands of concepts from external ontologies it imports.
To prevent name clashes, in actual usage the names of concepts, relations, and individuals are often name-spaced.
In this document, since we mostly talk about concepts from the SOMA ontologies,
the namespace will not be made explicit.
An exception will be made in some diagrams where we reference concepts defined in more basic ontologies,
such as those used to define the Ontology Web Language (OWL).
An example is a name such as \emph{xsd:double}; in this case, \emph{xsd} is the namespace.

% % % % % % % % % % % % % % % % %
% % % % % % % % % % % % % % % % %
\section{Scope} % Domain of Discourse
\label{sec:scope}
%\todo{Mihai: write about competency questions here. Probably would be good to provide a list of them that we intend to cover with the information contained in neems}

The broad scope of this work is to provide information about how robotic manipulation activities are represented, acquired, curated and published in the \ease system for episodic memories.
%Our work aims to provide knowledge modeling for robotic manipulation and autonomous robot control, such that data acquired through performing actions can be stored, interpreted, and used towards training and improving robot skills.
%The knowledge that is to be extracted from data must include 
We are in particular interested in
aspects of interaction forces and motion characteristics of objects participating in an action, since it is these physical and geometric considerations that are crucial for successfull action execution.
%determine whether, or to what degree, an action is successful.
The goal is to learn models from collections of recorded data semantically annotated through concepts defined in the \neem model.
The rich semantic annotations enable querying and filtering the data, such that a robot can formalize a learning problem for itself and curate its training data to be appropriate for it.
Information about how the data is collected, with what methods, from what agents, in which contexts, is important for this process, as machine learning techniques are sensitive to training data biases.
Note that in principle episodes can be stored of any agent performing any activity, and in actuality many of the NEEMs we expect to store will come from humans demonstrating how to perform a task.
NEEMs are therefore not simply intended as a kind of self-practice journal, but rather as a store of practical knowledge of a variety of agents, useful for a variety of autonomous, humanoid robots.

%Ideally, the knowledge we formalize in this collection of ontologies will allow a single, general control program to adapt itself, via reasoning, and generate adequate behavior in a large variety of settings, for different tasks and participating objects.
% NOTE: taken from SOMA paper
%The broad scope of our work is everyday object manipulation tasks in autonomous robot control, and in particular the motion and force characteristics of objects that interact with each other.
%The research question driving us is whether a single general control program can be written that can generate adequate behavior in many different contexts: for different tasks, objects, and environments.
%\todo{somebody needs to rephrase this! it is copied from SOMA paper}

The kinds of knowledge a robot needs for competent performance of its tasks are varied. Usually, knowledge modelling in robotics and AI has focused on a symbolic level, of actions treated as black boxes that relate to a larger plan by means of their preconditions and effects. Actions are also very underspecified when described in spoken commands. This abstract level of description however is insufficient; the physical details of the actions matter. For example, the angle and speed with which a pitcher is moved, and the amount of liquid in it, determines whether there will be spillage. A robot needs to choose appropriate parameters for its actions, and infer these parameters when they are left unspecified in a command.
% NOTE: taken from SOMA paper
%One of the challenges is that, using such a general plan, the agent needs to fill the knowledge gaps between abstract instructions included in the plan and the realization of context specific behavior. That is, for example, the many ways of how humans perform a pouring task depending on the source from which is poured, the destination, and the substance that is to be poured.
%Another challenge is that object manipulation tasks may fail if the agent does not perform the motions competently and well. This is caused by the agent choosing inappropriate parametrization of its control-level functions.
%\todo{somebody needs to rephrase this! it is copied from SOMA paper}

Such inference requires the robotic agent to be equipped with common-sense and intuitive physics knowledge, as well as an abstract task and object model, and knowledge of how to apply these models in a given situation. SOMA is an attempt to support each of these requirements. A brief list of some of the over-arching competency questions follows.

%%\begin{itemize}
%%    \item \emph{How are actions conceptualized?} What is an Action, how does it relate to other concepts an Agent might have about the world? What is the purpose of an Action?
%%    \item \emph{How are possibilities for action formalized?} How does an agent model what is made available by the environment? How does an agent model what it is itself capable of?
%%    \item \emph{What is the structure of an Action?} How do several actions make up another? What objects participate in an action and with what roles?
%%    \item \emph{How are qualitative and quantitative features of the world represented?} What is the parameter set of an action? What regions can values for these parameters occupy? What is a good parameterization and how can one be found?
%%\end{itemize}

\begin{itemize}
    \item \emph{How are actions conceptualized?} What is an action, how does it relate to other concepts an agent might have about the world? What is the purpose of an action?
    \item \emph{What is the structure of an action?} How do several actions make up another? What objects participate in an action and with what roles?
    \item \emph{How are qualitative and quantitative features of the world represented?} What is the parameter set of an action? What regions can values for these parameters occupy? What is a good parameterization and how can one be found?
    \item \emph{How are the physical interactions that underlie an action described?} What are the involved forces, and how are they parameterized? What are relevant qualitative, and thus more general, descriptors for interactions, such as balance, blockage, compulsion? How are qualitative aspects of interaction grounded in quantitative physical phenomena?
    \item \emph{How are objects conceptualized?} What roles can an object play? What actions can it take part in? What kinds of objects are necessary for an action?
    \item \emph{How is an action recorded and described?} What is the relevant data to capture how an action unfolded? What are the relevant pieces of contextual information for describing an action that has actually occurred? What was the outcome of the action, in particular, to what extent did it match the goal?
    \item \emph{How is a learning problem formalized?} What is the optimization goal? What assumptions were in effect when collecting the training data? What sort of influence might biases have upon the learned model? What should be essential features that a learned model should use? What would be sanity checks on the learned model to verify it does not abuse spurious correlations?
\end{itemize}
\todo{DB: could we add some motion/force questions here?}
\todo{DB: could we add some questions that require a ML model learned from neems?}

% NOTE: taken from SOMA paper
%The employment of a general plan thus requires an abstract task and object model, and a mechanism to apply this abstract knowledge in situational context.
%To achieve this, an agent needs to be equipped with the necessary common-sense and intuitive physics knowledge, which is what SOMA\todo{The acronym is no where defined} attempts.
%\todo{somebody needs to rephrase this! it is copied from SOMA paper}

% % % % % % % % % % % % % % % % %
% % % % % % % % % % % % % % % % %
\section{Overview}
\label{sec:overview}

% paragraph about the intention behind this document
\neems are the central data structures that link research results of various sub-areas within the collaborative research center \ease.
\ease is an interdisciplinary institution headed by leading researchers in the fields robotics, human cognition, formal logics, and linguistics.\todo{is this complete?}
The overall goal is to make a robot more competent in performing everyday activities.
This is accomplished by equipping the robot with models learned over experiences represented as \neems.
The purpose of this document is to provide detailed information about the \ease system for episodic memories.
That is how \neems are
represented as knowledge bases linked to time-series data,
acquired through experimentation, observation or simulation,
stored on a centralized server, and
maintained as a dataset for the research community.
The architecture is shown in Figure~\ref{fig:architecture}, and will be summarized in the remainder of this section.

% paragraph about representation (neem background, narrative, experience)
At the core of the \ease system for episodic memories is the \neem data structure.
It is a heterogenous datastructure that contains data in different formats to represent different categories of information about everyday activities.
Each \neem is made of three parts: background (Chapter~\ref{ch:background}), narrative (Chapter~\ref{ch:narrative}) and experience (Chapter~\ref{ch:experience}).
The background represents physical activity context by characterizing the environment, and agents that play a role during the activity.
A single background may be shared in mutiple \neems.
The narrative is a representation of events that happened, their characterization and contextualization.
That is, for example, that an event occurred, what roles objects played during the event, how the event was carried out through motions and interactions, and what the reason of its occurrence is.
The narrative provides labels used to annotate the time-series data stored in the experience of the \neem.
This is done by associating the event time intervals to slices in the time-series database.
The experience data is used to capture some aspects of kinematics and dynamics of an activity, that is how objects moved, how they got into contact with each other, and how forces act upon objects.

% paragraph about neem hub -- purpose, potential, ... -- add some buzzwords from data science
\neems are stored on a publicly accessible infrastructure that we have called the \neemhub (Chapter~\ref{ch:neemhub}).
The \neemhub builds on top of common infrastructure used in data science to continuously update models learned from \neems.
Uploading a \neem requires to create a new data set on the \neemhub GitLab interface where users can provide documentation, usage examples, additional links and references for their \neem data set.
Once a user is satisfied with the state of the data set, it may be published.
This will make the data set accessible via the knowledge service \openease where users may search for data sets given some keywords, download the data set, or investigate it in an interactive environment.

% paragraph about acquisition (robot, vr) -- what do they have in common? what is different?
As \ease is an interdisciplinary effort, there are also different modalities under which \neems can be acquired.
We haved developed multiple acquisition infrastructures that support researchers from different domains to acquire \neems (Chapter~\ref{ch:acquisition}).
This is, first of all, an interface that integrates with a robot control system either in a simulated or real-world scenario where the robot senses its surroundings, and executes specific plans through motions of its body and interactions with its environment.
A second acquisition interface integrates with simulated virtual reality environments in which humans perform everyday activities.
In this case, the intentions are not certainly known because even when told to do something specific, a human may do some unrelated experimentation in the virtual reality.
The state of the environment including force characteristics can, however, fully be monitored.\todo{DB: add ELAN section? Seba: Unless Mihai is familiar with that system, lets skip it for now.}



\chapter{Requirements}
A good idea would be to have here a table with all requirements which we want to achieve with NEEMs.
I assume it will be a nice overview for the reader(partner), if they will open the document, they can see right away what feature is supported by the current NEEM version, which are planned for the next version and which will be added in the late future. We can create the requirement table based on the EASE proposal.\todo{Seba: Good idea but I would wait until the whole document is finished.}

\chapter{Representation}
\label{chap:represenation}

% % % % % % % % % % % % % % % % % % % % % % % %
% % % Prelude
% % % % % % % % % % % % % % % % % % % % % % % %
\newcommand{\givenODPNAME}{}
\newcommand{\givenODPINTENT}{}
\newcommand{\givenODPDEFINEDIN}{}
\newcommand{\givenODPDESCRIPTION}{}
\newcommand{\givenODPGRAPHIC}{}
\newcommand{\givenODPDOMAIN}{}
\newcommand{\givenODPQUESTION}{}
\newcommand{\ODPINTENT}[1]     {\renewcommand{\givenODPINTENT}{#1}}
\newcommand{\ODPDEFINEDIN}[1]  {\renewcommand{\givenODPDEFINEDIN}{#1}}
\newcommand{\ODPDESCRIPTION}[1]{\renewcommand{\givenODPDESCRIPTION}{#1}}
\newcommand{\ODPGRAPHIC}[1]    {\renewcommand{\givenODPGRAPHIC}{#1}}
\newcommand{\ODPDOMAIN}[1]     {\renewcommand{\givenODPDOMAIN}{#1}}
\newcommand{\ODPQUESTION}[1]   {\renewcommand{\givenODPQUESTION}{#1}}
\newcommand{\OPDinit}{
  \renewcommand{\givenODPINTENT}{REQUIRED!}
  \renewcommand{\givenODPDEFINEDIN}{REQUIRED!}
  \renewcommand{\givenODPDESCRIPTION}{REQUIRED!}
  \renewcommand{\givenODPGRAPHIC}{REQUIRED!}
  \renewcommand{\givenODPQUESTION}{}
  \renewcommand{\givenODPDOMAIN}{}
  \renewcommand{\labelitemi}{$\mathbf{\sqsubseteq}$}
}

\newenvironment{owlclass}[2][,] {
  \begin{minipage}{5.0cm}
  \begin{center}
  \texttt{\bf#2} \\[-0.2cm]
  \par\noindent\rule{\textwidth}{0.4pt}
  \vspace{-0.6cm}
  \begin{itemize}[#1]
  \raggedright} {
  % % % % % %
  \end{itemize}
  \end{center}
  \end{minipage}
}

\newenvironment{ODP}[1]{
\OPDinit
\renewcommand{\givenODPNAME}{#1}
}{
\givenODPDESCRIPTION
\begin{figure}[htb]
\begin{minipage}{0.45\textwidth}
\begin{tabular}{ p{1.8cm} p{3.2cm} }
\toprule
% {\it\bf Name}                 & \emph{\givenODPNAME} \\
{\it\bf Intent}               & \givenODPINTENT \\
{\it\bf Domains}              & \givenODPDOMAIN \\
{\it\bf Competency Questions} & \givenODPQUESTION \\
{\it\bf Defined in}           & \givenODPDEFINEDIN \\
\bottomrule
\end{tabular}
\end{minipage}
\begin{minipage}{0.55\textwidth}
\begin{center}
\givenODPGRAPHIC
\end{center}
\end{minipage}
\caption{The \emph{\givenODPNAME} ODP.}
\end{figure}
}

\tikzset{owlclass/.style={draw=blue!40,fill=blue!20,rounded corners}}

% % % % % % % % % % % % % % % % %
\section{NEEM-Narrative}
\label{ch:narrative}

% NOTE: taken from KnowRob2.0
% One of the most powerful components of the human memory system is the
% episodic memory. 
When somebody talks about the deciding goal in the
last soccer world championship many of us can ``replay'' the episode
in our ``mind's eye''. %, or if somebody talks about a cross court hit
%in tennis we can recall how such a hit feels in the arm.
The memory
mechanism that allows us to recall these very detailed pieces of
information from abstract descriptions is our episodic memory.
Episodic memory is powerful because it allows us to remember special
experiences we had. It can also serve as a ``repository'' from which
we learn general knowledge.

% NOTE: taken from KnowRob2.0
EASE integrates episodic memories
deeply into the knowledge acquisition, representation, and processing
system. Whenever a robotic agent performs, observes, prospects, and
reads about an activity, it creates an episodic memory. An episodic
memory is best understood as a video that the agent makes of the
ongoing activity 
%in its inner world knowledge base
coupled with a very
detailed story about the actions, motions, their purposes, effects,
the behavior they generate, the images that are captured, etc.

% % % % % % % % % % % % % % % % %
% % % % % % % % % % % % % % % % %
\subsection{Scope} % Domain of Discourse
\label{sec:narrative:scope}

% NOTE: taken from SOMA paper
The broad scope of our work is everyday object manipulation tasks in autonomous robot control, and in particular the motion and force characteristics of objects that interact with each other.
The research question driving us is whether a single general control program can be written that can generate adequate behavior in many different contexts: for different tasks, objects, and environments.

% NOTE: taken from SOMA paper
One of the challenges is that, using such a general plan, the agent needs to fill the knowledge gaps between abstract instructions included in the plan and the realization of context specific behavior. That is, for example, the many ways of how humans perform a pouring task depending on the source from which is poured, the destination, and the substance that is to be poured.
Another challenge is that object manipulation tasks may fail if the agent does not perform the motions competently and well. This is caused by the agent choosing inappropriate parametrization of its control-level functions.

% NOTE: taken from SOMA paper
The employment of a general plan thus requires an abstract task and object model, and a mechanism to apply this abstract knowledge in situational context.
To achieve this, an agent needs to be equipped with the necessary common-sense and intuitive physics knowledge, which is what SOMA attempts.

\textbf{TODO: write about competency questions}

% % % % % % % % % % % % % % % % %
% % % % % % % % % % % % % % % % %
\subsection{Foundational Commitments}
\label{sec:narrative:commitments}

% NOTE: taken from SOMA paper
We decided to base our model on the DOLCE+DnS Ultralite (DUL) foundational framework~\cite{DOLCE2003}.
This decision is greatly motivated by their underlying ontological commitments.
%The decision to base our model on the DOLCE+DnS Ultralite foundational framework, is greatly motivated by their underlying ontological commitments.
Firstly, DUL is not a revisionary model, but seeks to express stands that shape human cognition. Furthermore it assumes a reductionist approach -- rather than capturing, for example, the flexibility of our usage of objects via multiple inheritance in a multiplicative manner, we commit to a reduced {\it ground} classification and use a {\it descriptive} approach for handling this flexibility. For this a primary branch of the ontology represents the ground {\bf physical model}, e.g. objects and actions, while a secondary branch represents the {\bf social model}, e.g. roles and tasks. All entities in the social branch would not exist without humans, i.e. they constitute social objects that represent concepts about or descriptions of ground elements. 

% NOTE: taken from SOMA paper
Every axiomatization in the physical branch can, therefore, be regarded as expressing some physical context whereas axiomatizations in the descriptive social branch are used to express social contexts. A set of dedicated relations is provided that connect both branches. For example, as detailed in Section~\ref{subsec:roles}, the relation \emph{classifies} connects ground objects, e.g. a hammer, with the roles they can play, i.e. potential classifications. Thus, we can state that a hammer can in some context be conceptualized as a murder weapon, a paper weight or a door stopper. Nevertheless, neither its ground ontological classification as a tool will change nor will hammers be subsumed as kinds of door stoppers, paper weights or weapons via multiple inheritance. Following a quick overview of the central modules of SOMA where these commitments apply, we will provide detailed examples of where and how our commitments apply in Sections~\ref{sec:perdurants} and~\ref{sec:endurants}.

% % % % % % % % % % % % % % % % %
% % % % % % % % % % % % % % % % %
\subsection{Taxonomy}
\label{sec:narrative:taxonomy}

\textbf{TODO: provide a broad overview about types in SOMA, what branches the ontology has. Show a Figure, maybe use this Figure to indicate in subsections at which branch in the taxonomy we are?}

% % % % % % % % % % % % % % % % %
% % % % % % % % % % % % % % % % %

\subsection{Actions}

\todo{Describe ACT, PROC, STATE from the SOMA paper}

\subsubsection{Ontology Design Patterns (ODP)}

\newpage
\subsubsection{Process vs. Action ODP}
\begin{ODP}{Process vs. Action}
\ODPDESCRIPTION{An Action is an Event with at least one Agent participant, such that this Agent has a Task, often defined by a Plan or Workflow, which it executes through the Action. A Process is an Event for which no such commitments have been made. In DUL, these classes are not disjoint, allowing a particular event individual to be classified as either, depending on whether we care to record an agent and its goals or not. In EASE, we use Process as a top-level class for events with no agentive participant.}
\ODPINTENT{To represent the intentional and agentive structure-- or lack thereof-- behind Events.}
\ODPDOMAIN{
  \texttt{Event classification},
  \texttt{Event narratives}}
\ODPDEFINEDIN{DUL.owl}
\ODPQUESTION{
  \emph{Is there anyone responsible for the event?}
  \emph{What are they trying to do?}
  \emph{How did an event unfold?}}
\ODPGRAPHIC{
\begin{tikzpicture}
 \node[owlclass] (ACTION) {
 \begin{owlclass}{ACTION}
  \item \texttt{Event}
  \item $(\exists \emph{hasParticipant}.\texttt{Agent})$
 \end{owlclass}
 };
 \node[owlclass,below=0.6cm of ACTION] (PROCESS) {
 \begin{owlclass}{Process}
  \item \texttt{Event}
 \end{owlclass}
 };
\end{tikzpicture}
}
\end{ODP}

\newpage

\subsection{Participants}

\todo{Describe the OBJ Module of Soma}

\subsubsection{Ontology Design Patterns (ODP)}

\todo{Objects are participants, Introduction}

\subsection{Situations}

\todo{Describe the EXEC Module of Soma}

\subsubsection{Ontology Design Patterns (ODP)}


% In version \neemversion the \neemnar consists the belief state and action task hierarchy. 
% In the following sections we will describe how the belief state and action task hierarchy are represented.
% An concrete example for a logged \neemnar will be given in Chapter \ref{ch:example}.

% \subsection{Belief State}
% 
In EASE, we investigate everyday manipulation activities.
These involve the interaction with physical objects such as
grasping an object and putting it somewhere,
or taking a tool and applying it on an object to achieve a certain effect.
To tell a comprehensive story about its activity,
an agent needs to memorize the relation of action to
physical objects that are salient during the action.

One requirement for this is that beliefs of the agent about the existence
of physical objects must be maintained by the system.
In some cases, when the existence is known in advance,
it is sufficient to supply this knowledge to the agent through
a static ontology holding facts about existence of physical objects
in the environment of the agent.
This is, for example, useful to supply knowledge about a static
environment such as a kitchen with fixed set of appliances.
In more dynamic set-ups, however, the beliefs about the existence
of physical objects must be formed through specialized perception
methods, or through logical reasoning.
This is, for example, the case when the particular set of objects
contained in a drawer are unknown in advance.
We consider this type of information as part of the NEEM narrative
to keep these beliefs persistent, and to allow referring to
the perceived objects in action descriptions.

Episodic memories capture information about temporal situations
during which the beliefs of the agent evolve according to
perception, action, and logical inference.
NEEMs capture this evolution of beliefs.
As such, revised or false beliefs are not deleted but kept
persistently as part of the NEEM narrative.
This is to allow for deeper analysis of the agent performance,
and to provide richer input for learning algorithms.
This is realized through a 4D ontology that we describe at
the end of this section.

%%%%%%%%%%%%%%%%%%%%%%%%%%%%%
%%%%%%%%%%%%%%%%%%%%%%%%%%%%%
%%%%%%%%%%%%%%%%%%%%%%%%%%%%%
%%%%%%%%%%%%%%%%%%%%%%%%%%%%%
\subsection{Tangible Objects}
To refer to objects in action descriptions only the name of the 
object must be known to the NEEM acquisition system.
\knowrob comes with a rather comprehensive object type system,
with about XXX different object classes~\cite{knowrob-ontology}.
The type system is represented as part of the TBOX of the knowledge base.
Perceived objects are represented as instance of one of the object types
provided by the \knowrob system.
This type of information is part of the ABOX of the knowledge base.
The name of an object is usually the name of the class followed
by a underscore character and a 8-digit hash,
but could potentially be any unique name string.

\todo{introduce base class of objects, give some details about taxonomy}

In systems relying on sensor information and statistical models one has to deal
with the uncertainty coming from the use of these information sources.
This also affects beliefs about the identity of objects.
The problem of identity resolution can trivially be approached
by computing euclidean distance to known objects.
If the distance is below a certain threshold, it can often be assumed that it is the same object.
Background knowledge can be used to find better estimates for object identity:
An object is inside the gripper as long as it is grasped,
pulled down by gravity to the plane below when the grasp is released again,
and so forth.
Also, the identity of objects is often not so important for decision making,
but rather the properties of the object.
For successfully preparing a slice of bread, for example, any butter will do.
But if there are two butter packages and one is not yet opened one would rather
use the opened one to avoid it getting rancid.
When acquiring NEEMs one has to deal with this identity resolution problem
to allow for referring to objects in the narrative part of the episodic memory.

In the remainder of this section we describe the fundamental binary predicates to
describe physical objects in the belief state.

\begin{description}
\item[\textbf{rdf:type}] 
The most fundamental assertion about an object next to its name is its type.
Types are asserted through the \emph{rdf:type}
property with one of the object types as value.
Ontologies are subsumption hierarchies that derive 
knowledge about more specific concepts from more general concepts.
This allows, for example, to describe the relation between object and action
on a general level that applies to a larger class of objects and actions.
It is further possible to assert multiple types from different branches
of the type hierarchy for a single object.
For example, some mobile phone may also be used as projector while other mobile phones,
without the projector hardware, should not be classified as projector.
%%%%%%%%%%%%%%%%%%%%%%%%%%%%%
%%%%%%%%%%%%%%%%%%%%%%%%%%%%%
\item[\textbf{physicalParts}]
$Whole\ \emph{physicalParts}\ Part$ means
that $Part$ is tangible and one of the distinct parts of the
more complex tangible object $Whole$.
Manipulation is often about putting together parts to form a bigger whole.
During table setting, for example, objects are arranged such that they form
place settings for the different participants.
The objects are placed intentionally such that humans can easily reach them from their seat,
see which objects belong to which place setting, and infer what objects belong to whom.
Another example are assembly tasks during which an agent tries
to put together some mechanical parts using screwing connections, snap-in connection,
and so on to create an assembled product from scattered pieces available.
NEEMs can also represent this partonomy information through dedicated predicates
linking parts to the bigger whole.
In particular, the \ease ontology defines the following sub-properties of \emph{physicalParts}:
\emph{dangerousParts} (e.g., the blade of a knife),
\emph{electricalParts} (i.e., parts that need electrical current to work),
\emph{mealComponents} (i.e., the ingredients of a meal).
%%%%%%%%%%%%%%%%%%%%%%%%%%%%%
%%%%%%%%%%%%%%%%%%%%%%%%%%%%%
\item[\textbf{frameName}] \dots
\end{description}

\todo{some other relevant predicates? shape, color, mesh, pose?}

%%%%%%%%%%%%%%%%%%%%%%%%%%%%%
%%%%%%%%%%%%%%%%%%%%%%%%%%%%%
%%%%%%%%%%%%%%%%%%%%%%%%%%%%%
%%%%%%%%%%%%%%%%%%%%%%%%%%%%%
\subsection{Kinematic Objects}
\begin{enumerate}
 \item \todo{object parts can be connected through joints}
 \item \todo{what types of joints exist?}
 \item \todo{what predicates can be used to describe them?}
\end{enumerate}

%%%%%%%%%%%%%%%%%%%%%%%%%%%%%
%%%%%%%%%%%%%%%%%%%%%%%%%%%%%
%%%%%%%%%%%%%%%%%%%%%%%%%%%%%
%%%%%%%%%%%%%%%%%%%%%%%%%%%%%
\subsection{Environment Maps}
Semantic maps are detailed representations of the environment of some agent.
These include geometrical and visual information about objects and appliances,
their functional decomposition, and their state.
Modern game engines reach an impressive level of detail, with individual
leaves falling down trees, etc.
In EASE, we try to reach this level of detail in our semantic map representations.
We do this to provide very comprehensive information about situated experiences
from which general knowledge can be learned.
We believe that this high level of detail will allow us to learn more robust
models and with less training data then would be possible with
a more abstracted representation of environments.

The \emph{SemanticMap} itself could be a \emph{Room}, a \emph{Building}, or some other type of connected
region in which some agent can do navigation and manipulation.
The ontology defines some more specific room classes including
\emph{Kitchen}, \emph{LivingRoom} and \emph{StoreRoom}.
This is useful, for example, to correlate objects with their likely storage room,
or to relate the rooms with common activities performed in them.

\begin{description}
%%%%%%%%%%%%%%%%%%%%%%%%%%%%%
%%%%%%%%%%%%%%%%%%%%%%%%%%%%%
\item[\textbf{describedInMap}]
$Object\ \emph{describedInMap}\ Map$ means \dots
\end{description}

\todo{also describe how rooms relate to buildings, etc.?}
\todo{something about the state?}


\subsection{Temporal Representation}
\dots

% \subsection{Action Hierarchy}
% \label{ch:narrative,sec:actionHierarchy}
% In this section, we will describe how an action task hierarchy is represented in the \neemnar. 
% Since we are using \cram on our robots our plans do not necessary generate a sequence of actions, instead they will generate rather an hierarchy of actions.
% During the plan execution we are logging all executed actions with its parameters and represent the hierarchy in \owl.
% The general idea of the model is that an action will be represented as an individual of the class \owlClass{knowrob:'Action'}.
% This individual can be a direct instance of the class \owlClass{knowrob:'Action'} or its subclass.
% \todo{Add all supported action sub classes}
% With the predicates \owlPredicate{subAction}, \owlPredicate{previousAction} and \owlPredicate{nextAction}, which all have as subject and object the type \owlClass{knowrob:'Action'}, we are able to represent the action hierarchy.
% In our understanding an logged action hierarchy in an \owl represents all actions which were executed during an experiment.
% Meaning, if we would extracted all actions from the \owl file and recreated the action tree, we would be able to analyze and reasoning about all executed actions during one specific experiment.

% In the next subsections we will describe the predicates and classes which we currently defined in the version \neemversion to log the actions which were executed by the robot during an experiment. 

% \subsection{Action Predicates}
% Every individual of the class \owlClass{knowrob:'Action'} class or its subclass which will be logged in the \owl file can be asserted with the following predicates (see Table \ref{table:action_task_predicates}).
% Some predicates in the table are marked as required.
% This means that if you are intending to upload your \owl file to \openease, every \owlClass{knowrob:'Action'} individual has to have the required properties asserted.
% %Otherwise the \owl file will be revoked from the server.\todo{We have to implement such checking in \openease.}
% The \openease server checks also if the objects of the predicates are associated with the correct class.

% \begin{table}[H]
% 	\begin{tabular}{| c | c | c | c |}
% 		\hline			
% 		\textbf{Subject} & \textbf{Predicate} & \textbf{Object}  & \textbf{Required} \\
% 		\hline
% 		Action & taskSuccess & xsd:boolean & Yes \\
% 		\hline
% 		Action & startTime & Timepoint & Yes \\
% 		\hline
% 		Action & endTime & Timepoint  & Yes \\
% 		\hline
% 		Action & subAction & Action & No \\
% 		\hline
% 		Action & nextAction & Action & No \\
% 		\hline
% 		Action & previousAction & Action & No \\
% 		\hline
% 	\end{tabular}
% 	\caption{Action Predicates}
% 	\label{table:action_task_predicates}
% \end{table}

% \begin{description}
% 	\item[\textbf{taskSuccess}] 
% 		This predicates points to data type \owlClass{xsd:boolean}.
% 		The value \textbf{true} represents that the action was executed successfully.
% 		If any errors occurred during the action execution, the data type will be set to \textbf{false}.
% 		\footnote{In the NEEM version \neemversion we do not log the exact error which happened during action.}
% 	\item[\textbf{startTime}]
% 		The \owlPredicate{startTime} represents when the action started.
% 		Instead of representing the startTime as a data point, we are creating an instance of the class \owlClass{Timepoint}.
% 		The implemenation is done in that way because it made writing prolog queries much more convenient. 
% 		The name of the individual is representing the exact time when the action started e.g.\ \textit{timepoint\_1523878415038090} which can be understood that the action started 1523878415038090 microseconds after 00:00:00 UTC, Thursday, 1 January 1970.
% 		We are using the Unix time to represent a time point \cite{matthew2011beginning}.
% 		However, we are considering to measure the time in microseconds.
% 		The reason for this decision is that we also want to create NEEMs in simulation.
% 		However, tasks in simulation can be executed so fast that logging in microseconds allowed to measure the performance of the task executions.
% 		Also the measurement in microseconds allowed to differentiate the running time between tasks.
% 	\item[\textbf{endTime}] 
% 		This predicate represents when the action ended.
% 		More information about how we log time points is described in the predicate description \owlPredicate{startTime}.
% 	\item[\textbf{subAction}]
% 		The predicate \owlPredicate{subAction} allows to create parent-child relation between two tasks. In the context of this predicate, the subject is the parent action and the object is the child action.
% 		It is possible that an \owlClass{Action} instance can have multiple \owlPredicate{subAction} predicates which point each to a single child action.
% 	\item[\textbf{nextAction}]
% 		To be able to create an sequential order of actions which where executed on the same hierarchy level, we defined the \owlPredicate{nextAction} predicate.
% 		The subject represents the action which was started first and points to the next sibling actions.
% 		\footnote{In the NEEM version \neemversion we are not differentiate if actions were executed in sequence or in parallel.}
% 	\item[\textbf{previousAction}]
% 		Like in \owlPredicate{previousAction} this predicate is created to create an sequential order between siblings tasks.
% 		However in this case, \owlPredicate{previousAction} connects an \owlClass{Action} instance with the sibling action which was performed previously.
% \end{description}


% \subsection{Action Parameter Predicates}
% 	\label{sec:actionParameterPredicactes}
% 	In general, actions have to have parameters which have to be asserted.
% 	Based on those assertions \cram is able to infer how to perform the task.
% 	For instance, a grasping task will be executed differently when the target object is a spoon compared to when the target object is a bottle.
% 	To understand the logged behavior of the robot better, we are logging to each action the corresponding parameters.
% 	For \cram we are using for our actions a set of predefined parameters.
% 	For example, when an action requires an object every \cram action has parameter name \textit{object} asserted.
% 	During the logging process, we are creating based on the parameter's value an instance of the class \owlClass{Object} and connect this instance with the \owlPredicate{objectActedOn}.
% 	\todo{Ref to belief state when object description is done}
% 	This implementation allows us to use the logger for new actions without the need to extend the logger.
% 	As long the \cram actions will use the predefined parameter names all parameters will be logged without the need to extend the logger.
% 	All predefined parameters are represented by a separated predicates which point to the parameter value which is an individual of the corresponding \owl class.
% 	Table \ref{table:action_parameter_predicates} shows the current parameter predicates which are supported by our NEEM representation.
% 	The design of the action parameter predicates is based on the work with our \pr.
% 	Therefore in the NEEM \neemversion it might be possible that the action parameters cannot be used by everyone in this state.
	 
% \begin{table}[H]
% \begin{tabular}{| c | c | c |}
% 	\hline			
% 	\textbf{Subject} & \textbf{Predicate} & \textbf{Object} \\
% 	\hline			
% 	Action & effort & qudt\#NewtonMeter \\
% 	\hline
%     Action & position & Float \\
%     \hline
% 	Action & arm & Pr2\#Pr2RightArm \\
% 	\hline
% 	Action & bodyPartsUsed & Pr2\#Pr2RightGripper \\
% 	\hline
% 	Action & goalLocation & Pose or Connected Space Region \\
% 	\hline
%     Action & objectActedOn & Object \\
% 	\hline
% 	Action & objectType & Object \\
% 	\hline
% \end{tabular}
% 	\caption{Action Parameter Predicates}
% 	\label{table:action_parameter_predicates}
% \end{table}

% \begin{description}
% 	\item[\textbf{effort}] 
% 		Effort is the grasping force in newton-meters.
% 		To model this we are using the \qudt ontology \footnote{http://qudt.org/}.
% 	\item[\textbf{position}]
% 		We are using this predicate to log the the goal position for the gripper of the \pr.
% 		The \pr accepts a joint angle in RAD to position its gripper.
% 		We decided to use a float data type to model the position to be able to represent more different types of position.
% 		For instance, our \boxy robot uses centimeters to position the gripper.
% 		Therefore with a float representation we are able to log from both robots the position parameter.
% 	\item[\textbf{arm}]
% 		With this predicate we want to log which arm was used to by the robot.
% 		The predicate points to an instance of the specific robot arm class.
% 		For instance, to model that the \pr used a right arm to grasp a bottle we are asserting the \owlPredicate{arm} predicate to an individual of the class \owlClass{Pr2RightArm}\footnote{http://knowrob.org/kb/PR2.owl}.
% 	\item[\textbf{bodyPartsUsed}]
% 		The \owlPredicate{bodyPartsUsed} predicate should represent which body part from the robot was used to perform the task.
% 		This predicate can be used \eg to represent that a gripper was used to perform the task.
% 		Like in \dots we also want to represent the body part as instance of the class which represents the body part.
% 	\item[\textbf{goalLocation}]
% 		This predicate can be used to represent the target or location parameter of an action.
% 		This predicate is suitable to represent \eg the target location of an \textit{Going} action.
% 		We are considering to different object types to log the goal location.
% 		Since our robots are build on \ros our robots can work with \textit{Poses} \footnote{http://docs.ros.org/api/geometry\_msgs/html/msg/Pose.html}.
% 		How we represent this \owl class is stated out in section \ref{sec:pose}.
		
% 		Since we are also using \cram, our robots can also encounter \textit{location designator} as goal locations\cite{beetz2010cram}.
% 		That means \cram can handle tasks like "Go to the kitchen counter".
% 		Since locations designators are more abstract we use the \owl class \owlClass{Connected Space Region} to log them.
% 		A more detailed description is given in section \ref{sec:connectedSpaceRegion}.
		
% 	\item[\textbf{objectActedOn}]
% 		With the predicate \owlClass{objectActedOn} we want to log which objects are given as parameter to the action.
% 		For instance, we can log which objects the robots looked for during a perception task and what objects he tried to grab.
% 	\item[\textbf{objectType}]
% 		The difference of \owlPredicate{objectType} compared to \owlPredicate{objectActedOn} is that \owlPredicate{objectType} define not a specific object but rather an object in general.
% 		For instance, if the robotic agent will get a task such as "grab the milk from the fridge".
% 		Given the task, the agent knows only that it has to grab a object of the type of milk from the fridge.
% 		So at this moment the robot does not know if the a milk box is actually in the fridge.
% 		Thereofre it has to recognize the milk first.
% 		If this was sucessfull then the robot will assoicated the following task the milk ID with the predicate \owlPredicate{objectActedOn} which the object will be the link to the speicifc object in the belieft state.
% \end{description}

% \subsection{Action Parameter Classes}
% As shown in section \ref{sec:actionParameterPredicactes} we are not only using data types to log the action parameter.
% In this section we are describing the \owl classes which we introduced to be able to log all parameters which we required to log during our experiments.

% \subsubsection{Pose}
% 	\label{sec:pose}
% 	This class is used to log coordinates given as parameter.
% 	Since our robots are using \ros we are logging poses.
% 	A pose consists from a Quaternion and a 3D vector.
% 	Since both together can only describe a Pose, both entities are required to be logged.
% 	\begin{table}[H]
% 		\begin{tabular}{| c | c | c | c |}
% 			\hline			
% 			\textbf{Subject} & \textbf{Predicate} & \textbf{Object} & \textbf{Required}\\
% 			\hline
% 			Pose & quaternion & String & Yes \\
% 			\hline
% 			Pose & translation & String & Yes \\			
% 			\hline
% 		\end{tabular}
% 		\caption{Pose Predicates}
% 		\label{table:pose_predicates}
% 	\end{table}
	
% 	\begin{description}
% 		\item[Quaternion] 
% 			%Quaternions are used to describe rotations in a three dimensionally space. \todo{Set reference to a Quaternions description}
% 			In the NEEM version \neemversion we are representing quaternions as a string.
% 			For instance, the quaternion $0.5 + 0.35i + 1j +0k$ will be represented as "0.5 0.35 1 0".
% 		\item[Translation]
% 			The \owlClass{Translation} class represents the 3D vector part of the pose.
% 			In the NEEM version \neemversion we are representing vectors as a string.
% 			For instance, the vector $\begin{bmatrix} -0.759, 1.19, 0.932 \end{bmatrix}^T$ will be represented as "-0.759 1.19 0.932".
			
			



% 	\end{description}
	
	
	
% \subsubsection{Connected Space Region}
% 	\label{sec:connectedSpaceRegion}
% 	Since we are using \cram our robots are allowed to use location designators to describe location parameters which will be resolved to pose \cite{beetz2010cram}
% 	The resolving process is divided in three stages:
% 		\begin{enumerate}
% 			\item Define a abstract location or general location. For instance, "Grab the milk from the fridge.".
% 			\item \cram will try out to resolve "fridge" to an entity of the semantic map.
% 			\item The last step is to resolve the entity from the semantic map into a pose with which the robot can actually work with.
% 		\end{enumerate}
	
% 	\begin{table}[H]
% 		\begin{tabular}{| c | c | c | c |}
% 			\hline			
% 			\textbf{Subject} & \textbf{Predicate} & \textbf{Object} & \textbf{Required}\\
% 			\hline
% 			Connected Space Region & onPhysical & iai-kitchen & No \\
% 			\hline
% 		\end{tabular}
% 		\caption{Connected Space Region Predicates}
% 		\label{table:connected_space_region_predicates}
% 	\end{table}
	
% 	\begin{description}
% 		\item[onPhysical] 
% 			To be able to represent the resolution to an entity of the semantic map we defined the \owlPredicate{onPhysical} predicate.
% 			This predicate points to an instance of semantic map.
% 			For instance, to grasp an object from the kitchen counter in our kitchen we link the \owlClass{ConnectedSpaceRegion} instance to an individual of our semantic mal in this case \textit{knowrob:iai\_kitchen\_sink\_area\_counter\_top}.
% 	\end{description}

% \subsubsection{Timepoint}
% 	In the NEEM version \neemversion the \owlClass{Timepoint} does not have any predicates.
% 	An instance of the \owlClass{Timepoint} class defines a moment in time which is represented in microseconds.
% 	The exact timestamp in microseconds is represent in the name of the instance.
% 	For example, the individual with the name "timepoint\_1523878419.243441" represents a timepoint which is 1523878419243441 microseconds after 00:00:00 UTC, Thursday, 1 January 1970.
% 	We are using the Unix time to represent a time point\cite{matthew2011beginning}.
% 	This Timepoint represenation makes the prolog quering much more easier.

% 
% % % % % % % % % % % % % % % % % % % % % % % %
% % % Prelude
% % % % % % % % % % % % % % % % % % % % % % % %
\newcommand{\givenODPNAME}{}
\newcommand{\givenODPINTENT}{}
\newcommand{\givenODPDEFINEDIN}{}
\newcommand{\givenODPDESCRIPTION}{}
\newcommand{\givenODPGRAPHIC}{}
\newcommand{\givenODPDOMAIN}{}
\newcommand{\givenODPQUESTION}{}
\newcommand{\ODPINTENT}[1]     {\renewcommand{\givenODPINTENT}{#1}}
\newcommand{\ODPDEFINEDIN}[1]  {\renewcommand{\givenODPDEFINEDIN}{#1}}
\newcommand{\ODPDESCRIPTION}[1]{\renewcommand{\givenODPDESCRIPTION}{#1}}
\newcommand{\ODPGRAPHIC}[1]    {\renewcommand{\givenODPGRAPHIC}{#1}}
\newcommand{\ODPDOMAIN}[1]     {\renewcommand{\givenODPDOMAIN}{#1}}
\newcommand{\ODPQUESTION}[1]   {\renewcommand{\givenODPQUESTION}{#1}}
\newcommand{\OPDinit}{
  \renewcommand{\givenODPINTENT}{REQUIRED!}
  \renewcommand{\givenODPDEFINEDIN}{REQUIRED!}
  \renewcommand{\givenODPDESCRIPTION}{REQUIRED!}
  \renewcommand{\givenODPGRAPHIC}{REQUIRED!}
  \renewcommand{\givenODPQUESTION}{}
  \renewcommand{\givenODPDOMAIN}{}
  \renewcommand{\labelitemi}{$\mathbf{\sqsubseteq}$}
}

\newenvironment{owlclass}[2][,] {
  \begin{minipage}{5.0cm}
  \begin{center}
  \texttt{\bf#2} \\[-0.2cm]
  \par\noindent\rule{\textwidth}{0.4pt}
  \vspace{-0.6cm}
  \begin{itemize}[#1]
  \raggedright} {
  % % % % % %
  \end{itemize}
  \end{center}
  \end{minipage}
}

\newenvironment{ODP}[1]{
\OPDinit
\renewcommand{\givenODPNAME}{#1}
}{
\givenODPDESCRIPTION
\begin{figure}[htb]
\begin{minipage}{0.45\textwidth}
\begin{tabular}{ p{1.8cm} p{3.2cm} }
\toprule
% {\it\bf Name}                 & \emph{\givenODPNAME} \\
{\it\bf Intent}               & \givenODPINTENT \\
{\it\bf Domains}              & \givenODPDOMAIN \\
{\it\bf Competency Questions} & \givenODPQUESTION \\
{\it\bf Defined in}           & \givenODPDEFINEDIN \\
\bottomrule
\end{tabular}
\end{minipage}
\begin{minipage}{0.55\textwidth}
\begin{center}
\givenODPGRAPHIC
\end{center}
\end{minipage}
\caption{The \emph{\givenODPNAME} ODP.}
\end{figure}
}

\tikzset{owlclass/.style={draw=blue!40,fill=blue!20,rounded corners}}

% % % % % % % % % % % % % % % % % % % % % % % %
% % % % % % % % % % % % % % % % % % % % % % % %
% % % % % % % % % % % % % % % % % % % % % % % %
\chapter{Ontology Design Patterns (ODPs)}

% % % % % % % % % % % % % % % % % % % % % % % %
% \section{Object Roles}

% % % % % % % % % % % % % % % % % % % % % % % %
% % % % % % % % % % % % % % % % % % % % % % % %
% \section{Quality and Quality Regions}

% % % % % % % % % % % % % % % % % % % % % % % %
% % % % % % % % % % % % % % % % % % % % % % % %
% \section{Designed Artifacts}

% % % % % % % % % % % % % % % % % % % % % % % %
% % % % % % % % % % % % % % % % % % % % % % % %
\section{Task Execution ODP}
\begin{ODP}{Task Execution}
\ODPDESCRIPTION{
This ODP allows to make assertions on roles played by agents
without involving the agents that play that roles, and vice versa.
It allows to express neither the context type in which tasks are defined,
not the particular context in which the action is carried out.
Moreover, it does not allow to express the time at which
the task is executed through the action
(for actions that do not solely execute that certain task).}
\ODPINTENT{To represent actions through which tasks are executed. }
\ODPDOMAIN{
  \texttt{Organization},
  \texttt{Management},
  \texttt{Scheduling},
  \texttt{Workflow}}
\ODPDEFINEDIN{DUL.owl}
\ODPQUESTION{
  \emph{Which task is executed through this action?}
  \emph{What actions can execute that task?}}
\ODPGRAPHIC{
\begin{tikzpicture}
 \node[owlclass] (ACTION) {
 \begin{owlclass}{Action}
  \item $(\geq 1\ \emph{executesTask}.\texttt{Task})$
 \end{owlclass}
 };
 \node[owlclass,below=0.6cm of ACTION] (TASK) {
 \begin{owlclass}{Task}
  \item $(\forall \emph{isExecutedIn}.\texttt{Action})$
 \end{owlclass}
 };
 \draw (ACTION) edge[thick,-,dashed,blue!60] (TASK);
\end{tikzpicture}
}
\end{ODP}

\newpage
\section{Quality -- Region ODP}
\begin{ODP}{Quality -- Region}
\ODPDESCRIPTION{This ODP allows structuring information about the properties of an Entity. It distinguishes between dependent aspects of the Entity (things that cannot exist without the Entity itself existing), and the values that may be ascribed to those aspects. These values may be points in some Region. Note that a Region may be a finite set of discrete labels, allowing for ``qualitative'' descriptions, but more often a Region is some dimensional space allowing ``quantitative'' descriptions. A Region may contain a single point, in cases where the value of a Quality is known precisely.}
\ODPINTENT{To distinguish between an aspect of an Entity and a particular numerical description of it.}
\ODPDOMAIN{
  \texttt{Measurement},
  \texttt{Object representation},
  \texttt{Environment representation},
  \texttt{Execution status}}
\ODPDEFINEDIN{DUL.owl}
\ODPQUESTION{
  \emph{What qualities does an entity have?}
  \emph{What are possible values for a quality?}
  \emph{What is the actual value of a quality for a particular entity (at a particular time)?}}
\ODPGRAPHIC{
\begin{tikzpicture}
 \node[owlclass] (QUALITY) {
 \begin{owlclass}{Quality}
  \item $(\exists \emph{isQualityOf}.\texttt{Entity})$
  \item $(\exists \emph{hasRegion}.\texttt{Region})$
 \end{owlclass}
 };
 \node[owlclass,below=0.6cm of QUALITY] (REGION) {
 \begin{owlclass}{Region}
  \item $(\exists \emph{isRegionFor}.\texttt{Quality})$
 \end{owlclass}
 };
 \draw (QUALITY) edge[thick,-,dashed,blue!60] (REGION);
\end{tikzpicture}
}
\end{ODP}

\newpage
\section{Process vs. Action ODP}
\begin{ODP}{Process vs. Action}
\ODPDESCRIPTION{An Action is an Event with at least one Agent participant, such that this Agent has a Task, often defined by a Plan or Workflow, which it executes through the Action. A Process is an Event for which no such commitments have been made. In DUL, these classes are not disjoint, allowing a particular event individual to be classified as either, depending on whether we care to record an agent and its goals or not. In EASE, we use Process as a top-level class for events with no agentive participant.}
\ODPINTENT{To represent the intentional and agentive structure-- or lack thereof-- behind Events.}
\ODPDOMAIN{
  \texttt{Event classification},
  \texttt{Event narratives}}
\ODPDEFINEDIN{DUL.owl}
\ODPQUESTION{
  \emph{Is there anyone responsible for the event?}
  \emph{What are they trying to do?}
  \emph{How did an event unfold?}}
\ODPGRAPHIC{
\begin{tikzpicture}
 \node[owlclass] (Action) {
 \begin{owlclass}{Action}
  \item \texttt{Event}
  \item $(\exists \emph{hasParticipant}.\texttt{Agent})$
 \end{owlclass}
 };
 \node[owlclass,below=0.6cm of ACTION] (PROCESS) {
 \begin{owlclass}{Process}
  \item \texttt{Event}
 \end{owlclass}
 };
\end{tikzpicture}
}
\end{ODP}

\newpage
\section{Designed Artifact ODP}
\begin{ODP}{Designed Artifact}
\ODPDESCRIPTION{A DesignedArtifact is a physical object described by a Design. In EASE, Designs refer to the form, but also the function of an object. This allows us to say that an object is ``for'' a particular purpose, even though it might be used for something else instead. For example, a cup is a BeverageContainer but can be used as a Flowerpot. Designs form a hierarchy of specificity, for example $DesignMilkContainer \sqsubseteq DesignBeverageContainer \sqsubseteq DesginContainer \sqsubseteq Design$. The justification for this pattern is that the type of an object is rigid, but the roles it plays in events change. A naive taxonomy, without a notion similar to Design, cannot tackle the fact that objects are usable in several ways beyond the obvious; a hammer isn't always a hammer, sometimes it's a paperweight. On the other hand, a usable ontology of objects must take into account how human users refer to objects by their default use.}
\ODPINTENT{To explicate the intuitive classification human users would have of objects, based on their default uses.}
\ODPDOMAIN{
  \texttt{Object classification},
  \texttt{Event narratives}}
\ODPDEFINEDIN{DUL.owl, EASE.owl, EASE-middle.owl}
\ODPQUESTION{
  \emph{What sort of object is this?}
  \emph{What is the intended use of the object?}
  \emph{How did an event unfold?}}
\ODPGRAPHIC{
\begin{tikzpicture}
 \node[owlclass] (DESIGNEDARTIFACT) {
 \begin{owlclass}{DesignedArtifact}
  \item \texttt{PhysicalArtifact}
  \item $(\exists\emph{isDescribedBy}.\texttt{Design})$
 \end{owlclass}
 };
 \node[owlclass,below=0.6cm of DESIGNEDARTIFACT] (DESIGN) {
 \begin{owlclass}{Design}
  \item \texttt{Description}
 \end{owlclass}
 };
 \draw (DESIGNEDARTIFACT) edge[thick,-,dashed,blue!60] (DESIGN);
\end{tikzpicture}
}
\end{ODP}

\newpage
\section{Transient ODP}
\begin{ODP}{Transient}
\ODPDESCRIPTION{This pattern addresses the fact that objects change by undergoing/taking part in Processes. For example, the PancakeMix becomes, through Baking, a Pancake. However, the ontological status of the object while the process takes place is unclear: the object placed on the frying pan is not a Pancake until Baking finishes, but it's not PancakeMix either once it begins to coagulate. In the EASE approach, the object in-between such characterizations is a Transient. A Transient transitionsFrom an Object, and possibly transitionsTo an Object. It may also be the case that a Transient transitionsBack to an Object, to indicate that once the process completes, the same Object is restored; this would be the case for example for catalysts in chemistry, or a loaf of bread after slicing, if there is enough bread left.}
\ODPINTENT{Ontological classification for objects undergoing type changes.}
\ODPDOMAIN{
  \texttt{Object classification},
  \texttt{Event narratives}}
\ODPDEFINEDIN{EASE.owl, EASE-middle.owl}
\ODPQUESTION{
  \emph{What sort of object is this?}
  \emph{What objects ``went'' in the making of another?}
  \emph{Does an object preserve or restore its identity after change?}}
\ODPGRAPHIC{
\begin{tikzpicture}
 \node[owlclass] (TRANSIENT) {
 \begin{owlclass}{Transient}
  \item \texttt{Object}
  \item $(\exists\emph{transitionsFrom}.\texttt{Object})$
  \item $(\forall\emph{transitionsTo}.\texttt{Object})$
 \end{owlclass}
 };
 \node[owlclass,below=0.6cm of TRANSIENT] (TRANSITIONSBACK) {
 \begin{owlclass}{transitionsBack}
  \item \texttt{transitionsFrom}
  \item \texttt{transitionsTo}
 \end{owlclass}
 };
\end{tikzpicture}
}
\end{ODP}

\newpage
\section{States}
\begin{ODP}{State, Configuration, Gestallt}
\ODPDESCRIPTION{A state is a configuration of the world that is construed to be stable on its own. Outside disturbances may cause state transitions, and the settling into some other, self-stable configuration. A State is also characterized by a Description, that indicates things such as what kind of entities participate in the state, what relations might exist between them, what regions may be used by particular qualities of the participants. This Description is, in general, referred to as a Configuration, however some common examples are Goals-- describe desired states of the world--, Norms-- describe states that should be kept--, and Diagnoses-- describe a state that causes certain observable symptoms. States are classified by Gestallts.}
\ODPINTENT{Ontological representation for situations in the world that are cognitively construed as stable arrangements of entities.}
\ODPDOMAIN{
  \texttt{Event classification},
  \texttt{Event narratives}}
\ODPDEFINEDIN{EASE-STATE.owl}
\ODPQUESTION{
  \emph{What are stable arrangements?}
  \emph{What is meant by ``state'' of the world?}
  \emph{What characterizes a state?}}
\emph{Examples}
\begin{itemize}
  \item AssemblyConnection: two objects are in a rigid connection, such that the movement of one determines the movement of the other. In this case the characterizing Configuration for this State uses several Roles-- one for each part/geometric feature belonging to the connected objects-- and puts constraints on the relative positioning of these geometric features such that they interlock to produce the rigid connection.
  \item Contact: two objects are in mechanical contact. The characterizing Configuration uses two Roles, one for each participating object, and puts constraints on the Pose qualities of the participants: the poses should be such that the participants touch.
  \item FunctionalControl: an object restricts the movement of another, at least partially. The Configuration uses the Roles Item and Restrictor. More concrete examples are Containment: the Restrictor is a Container, and the Pose quality of the Item should use the region inside the Container; and Support: both Restrictor and Item are objects, placed in such a way that the Item does not move because of gravity.
  \item PhysicallyAccessible: the Configuration for this state uses the roles Item, a Container or Protector, and optionally an Accessor and a Task, and states that an Item is either placed in a Container or protected by a Protector, but the placement of the Item and Container is such that an Accessor may nevertheless reach the Item in order to perform a Task. For a more concrete example, a DoorOpen is a kind of PhysicallyAccessible where the Protector is a door, the Item is the inside of the room behind the door, the Accessor is some person and the Task is to walk into the inside of the room.
\end{itemize}
%\ODPGRAPHIC{
%\begin{tikzpicture}
% \node[owlclass] (TRANSIENT) {
% \begin{owlclass}{Transient}
%  \item \texttt{Object}
%  \item $(\exists\emph{transitionsFrom}.\texttt{Object})$
%  \item $(\forall\emph{transitionsTo}.\texttt{Object})$
% \end{owlclass}
% };
% \node[owlclass,below=0.6cm of TRANSIENT] (TRANSITIONSBACK) {
% \begin{owlclass}{transitionsBack}
%  \item \texttt{transitionsFrom}
%  \item \texttt{transitionsTo}
% \end{owlclass}
% };
%\end{tikzpicture}
%}
\end{ODP}

% % % % % % % % % % % % % % % % % % % % % % % %
% % % % % % % % % % % % % % % % % % % % % % % %
% \section{Workflows}

% % % % % % % % % % % % % % % % % % % % % % % %
% % % % % % % % % % % % % % % % % % % % % % % %
% \section{Action Phases}

% % % % % % % % % % % % % % % % % % % % % % % %
% % % % % % % % % % % % % % % % % % % % % % % %
% \section{Transients}


% \subsection{Knowledge Bases}

In this chapter we should describe all knowledge bases we are using to represent the NEEMs.

\subsubsection{IAI-Kitchen}
\subsubsection{Knowrob}
\subsubsection{PR2}
\subsubsection{Qudt}


\chapter{NEEM-Experience}

\neemexp captures low-level information about experienced activities
represented as time series data streams.
This data has often no or only unfeasible
lossless representation as facts in a knowledge base.
To make this data \emph{knowledgable}, procedural hooks
are defined in the ontology to compute symbols from the experience data,
and to embed these symbols in logic-based reasoning.

The data is stored in a NoSQL database using JSON documents.
Each individual type of data is stored in a separate collection
named according to the type of data stored in the collection.
When imported, the knowledge system stores the data in a
\mongodb\footnote{https://www.mongodb.com/} server, for which
the knowledge system implements a client for querying the data
during question answering.
The query cursor concept employed in \mongodb integrates
nicely with backtracking based search employed in the knowledge system.
It further scales well to large amount of data and can be distributed amongst
clusters through built-in automatic sharding.

The data in \neemexps is represented as time series
and indexed in time order.
The different experience data types need to define a dedicated
time key for computing the search index.

% the \neemexp consists only of \tf data \cite{tf}.
% When executing an experiment in projection, we are collecting \tf data from the map, objects and the robot.
% In real world experiments we log only the \tf data from the map and the robotic agent.
% We are storing the \tf data in a \mongodb\footnote{https://www.mongodb.com/}.
% In the next iteration of the document we will provide links where you can download the required tools for log \tf data for NEEMs.

The experience data in \neems has individual characteristics regarding
the format, compressed representation, and what symbols the 
knowledge system can abstract from the data.
In this chapter, we describe these aspects for the experience data types covered 
in \neem version \neemversion.

%%%%%%%%%%%%%%%%%%%%%%%%%%%%%%%%%
%%%%%%%%%%%%%%%%%%%%%%%%%%%%%%%%%
\section{Pose Data}

A robotic system typically has many mobile components arranged in a kinematic chain.
Each component in a kinematic chain has an associated named coordinate frame such as
world frame, base frame, gripper frame, head frame, etc.
Coordinate systems are always 3D, with \emph{x} forward, \emph{y} left, and \emph{z} up.
6 DOF relative poses are assigned to the different frames.
These are usually updated with about 10 Hz during movements, and
expressed relative to the 
parent in the kinematic chain to avoid updates when only the parent frame moves.
The transformation tree is rooted in the dedicated world frame node
(also often called map frame).

The data is used by the \ease knowledge system to answer questions such as:
\begin{itemize}
% questions taken from TF docu page
 \item Where was the head frame relative to the world frame, 5 seconds ago?
 \item What is the pose of the object in the gripper relative to the base?
 \item What is the pose of the base frame in the map frame? 
\end{itemize}

% \begin{center}
% \includegraphics[height=0.3\textwidth]{img/links-joints.png}
% \includegraphics[height=0.3\textwidth]{img/tf-frames.png}
% \end{center}

Pose data is saved in \mongodb collections named ``tf'', the format is described below.

%%%%%%%%%%%%%%%%%%%%%%%%%%%%%%%%%
%%%%%%%%%%%%%%%%%%%%%%%%%%%%%%%%%
\subsection{Format}
The pose data structure has
fields for encoding the translation and rotation of a coordinate frame.
The parent frame and time stamp of pose estimation
are stored in the \emph{header} field of the data structure.
The transform coordinate frame is assigned to the \emph{child\_frame\_id} field.

The data is stored in the DB collection as array.
Each array holding pose estimates for distinct frames
at the same time stamp.
For indexed search, it is important that each array
member has the same time stamp. \\

\def\arraystretch{1.1}%
\begin{figure}[htb]
\begin{center}\begin{tabular}{ >{\ttfamily}p{3.5cm} >{\ttfamily}p{2cm} p{5cm} }
\toprule
\bf Field   & \bf Type & \bf Description \\ \midrule
tf			& [dict]	& -- \\
\ \ header		& dict		& -- \\
\ \ \ \ seq		& uint32	& consecutively increasing ID \\
\ \ \ \ stamp		& time		& time stamp of this transform \\
\ \ \ \ frame\_id	& string	& parent coordinate frame of this transform \\
\ \ child\_frame\_id	& string	& coordinate frame of this transform \\
\ \ transform		& dict		& -- \\
\ \ \ \ translation	& dict		& -- \\
\ \ \ \ \ \ x		& float64	& \emph{x} axis translation \\
\ \ \ \ \ \ y		& float64	& \emph{y} axis translation \\
\ \ \ \ \ \ z		& float64	& \emph{z} axis translation \\
\ \ \ \ rotation	& dict		& -- \\
\ \ \ \ \ \ x		& float64	& \emph{x} component of quaternion \\
\ \ \ \ \ \ y		& float64	& \emph{y} component of quaternion \\
\ \ \ \ \ \ z		& float64	& \emph{z} component of quaternion \\
\ \ \ \ \ \ w		& float64	& \emph{w} component of quaternion \\
\bottomrule
\end{tabular}\end{center}
\caption{The pose data structure in the \ease system.}
\label{fig:pose_data}
\end{figure}

Note that static frames may be recorded at lower frequency --
about every two seconds.
This usually reduces the data size significantly.
At the moment, no other motion data compression,
such as motion JPEG, is supported.

% From ``Compression of Motion Capture Databases'' Okan Arikan
% The biggest goal of compression is creating a compressed rep-
% resentation of motion that is perceptually as close to the original
% motion as possible. As we will explore later in this paper, a small
% numerical error does not necessarily correspond to a perceptually
% close motion. We would like compression and decompression to be
% as quick as possible. In practice motion capture databases can be
% very big. Therefore another goal for compression and decompres-
% sion is to be able to process without holding the entire database in
% the memory, which may not be possible. Depending on the appli-
% cation we may want to “stream” the data so that the decompressor
% can decode incrementally. We may also want to be able to decode a
% piece of the database without having to decompress any other mo-
% tion

%%%%%%%%%%%%%%%%%%%%%%%%%%%%%%%%%
%%%%%%%%%%%%%%%%%%%%%%%%%%%%%%%%%
\subsection{Symbol Abstraction}
%%%%%%%%%%%%%%%%%%%%%%%%%%%%%%%%%
\begin{center}
\begin{tabular}{ >{\ttfamily\bf}p{3.5cm} >{\ttfamily}p{8.2cm} }
\toprule
Module  & knowrob\_objects\footnote{\url{https://github.com/knowrob/knowrob/tree/master/knowrob\_objects}} \\
Symbols & object pose \\
Implementation & Prolog \\
\bottomrule
\end{tabular}
\end{center}
\begin{description}
\item[\textbf{belief\_at(Object, [Parent,Child,Translation,Rotation], Instant)}]
This temporal predicate computes a Prolog-based pose representation (2nd argument)
for a named object (1st argument). Time is supplied to the predicate as time stamp
(3rd argument).
\end{description}

%%%%%%%%%%%%%%%%%%%%%%%%%%%%%%%%%
\begin{center}
\begin{tabular}{ >{\ttfamily\bf}p{3.5cm} >{\ttfamily}p{8.2cm} }
\toprule
Module  & comp\_spatial\footnote{\url{https://github.com/knowrob/knowrob/tree/master/comp\_spatial}} \\
Symbols & spatial relations \\
Implementation & Prolog \\
\bottomrule
\end{tabular}
\end{center}


\begin{description}
\item[\textbf{comp\_inCenterOf(Inner,Outer,Interval)}]
Check if \owlClass{Inner} is in the center of \owlClass{Outer}.
Currently does not take the orientation into account, only the position and dimension.
Computes the \owlPredicate{inCenterOf} relation.

\item[\textbf{comp\_inFrontOf(Front,Back,Interval)}]
Check if \owlClass{Front} is in front of \owlClass{Back}.
Currently does not take the orientation
into account, only the position and dimension.
Computes the \owlPredicate{inFrontOf-Generally} relation.

\item[\textbf{comp\_inContGeneric(Inner,Outer,Interval)}]
True iff the object \owlClass{Inner} is inside 
of the bounding box of container \owlClass{Outer} during
the specified interval.
Computes the \owlPredicate{in-ContGeneric} relation.

\item[\textbf{comp\_onPhysical(Top,Bottom,Interval)}]
Check if \owlClass{Top} is in the area of and above \owlClass{Bottom}.
Computes the \owlPredicate{on-Physical} relation

\item[\textbf{comp\_above(Top,Bottom,Interval)}]
Check if \owlClass{Top} is in the area of and above \owlClass{Bottom}.
Computes the \owlPredicate{above-Generally} relation.

\item[\textbf{comp\_below(Bottom,Top,Interval)}]
Check if \owlClass{Top} is in the area of and above \owlClass{Bottom}.
Computes the \owlPredicate{below-Generally} relation.

\item[\textbf{comp\_toTheLeftOf(Left,Right,Interval)}]
Check if \owlClass{Left} is to the left of \owlClass{Right}.
Currently does not take the orientation
into account, only the position and dimension.
Computes the \owlPredicate{toTheLeftOf} relation.

\item[\textbf{comp\_toTheRightOf(Right,Left,Interval)}]
Check if \owlClass{Left} is to the left of \owlClass{Right}.
Currently does not take the orientation
into account, only the position and dimension.
Computes the \owlPredicate{toTheRightOf} relation.
\end{description}


%%%%%%%%%%%%%%%%%%%%%%%%%%%%%%%%%
%%%%%%%%%%%%%%%%%%%%%%%%%%%%%%%%%
% \section{Image Data}
% \input{content/neem-experience/image-data}

%%%%%%%%%%%%%%%%%%%%%%%%%%%%%%%%%
%%%%%%%%%%%%%%%%%%%%%%%%%%%%%%%%%
% \section{Gaze Data}
% \input{content/neem-experience/gaze-data}

%%%%%%%%%%%%%%%%%%%%%%%%%%%%%%%%%
%%%%%%%%%%%%%%%%%%%%%%%%%%%%%%%%%
% \section{EEG Data}
% \input{content/neem-experience/eeg-data}

\chapter{NEEM-Acquisition}
\label{ch:acquisition}
\chapterauthor{S. Koralewski, A. Hawkin}

This chapter focuses on the acquisition process of \neems.
At first, we will provide the tools and procedures to acquire episodic memories from robots performing experiments.
The second section focuses on the \neem acquisition from virtual reality. 

Each section will contain an example \neem to provide insights how the representation, described in Chapter \ref{chap:represenation}, is utilized to represent performed activities by robots or by humans. 
In addition, each example \neem is available on the \neemhub for downloading.


\todo{Seba: Should we describe here the data structure of the neem ?}
%\subsection{Data}
%
%Your \neem will consist of at least 5 folders - \textit{annotations}, \textit{inferred}, \textit{roslog}, \textit{ros\_tf} and \textit{triples}.
%In the following, we will give an overview which information is contained in those folders:
%
%\todo{@Seba add proper descriptions}
%\begin{description}
%	\item[annotations] description
%	\item[inferred] description
%	\item[ros\_tf] description
%	\item[triples] description	
%\end{description}


\section{Narrative Enabled Episodic Memories for Robotic Agents}
\label{sec:robot-neem}
\lstset{style=lispcode}
This section focuses on describing how to generate \neems from experiments performed by the robot. 

\subsection{Prerequisite}

Before you are intending to generate your episodic memories, make sure you are familiar with the Cognitive Robot Abstract Machine (\cram)\footnote{\url{http://cram-system.org/cram}} system and installed it on your machine.
\cram is a cognitive-enabled planning framework which allows to design high-level plan for robots.\todo{@Seba: Add more detailed description}
This section requires that your robot plan is written in \cram to be able to generate \neems.
However, once you are familiar with our planning and logging components, you will be able to port those components to your preferred planning tool.

In addition to having a valid \cram plan, you will need the following software components to be installed:
\begin{itemize}
	\item A MongoDB server with at least version 3.4.10\footnote{\url{https://www.mongodb.com/}}
	\item \knowrob\footnote{\url{https://github.com/knowrob/knowrob}}
	\item \soma ontology\footnote{\url{https://github.com/ease-crc/soma}}
	\item \cram ontology\footnote{\url{https://github.com/ease-crc/cram\_knowledge}}
\end{itemize}


\subsection{Recording Narrative Enabled Episodic Memories}
Our recording mechanism  captures every executed \cram action and its parameter.
In addition, the logger puts the actions in relation to each other by creating a hierarchy which is described in Section \ref{ch:narrative,sec:actionHierarchy}.

Before you can begin to record your own \neems, you need to include the "cram-cloud-logger" package into your \cram plan.
After you included the package, you need to enable the logging via:


\begin{lstlisting}[language=lisp, caption=Enabling \neem Logging in a \cram plan]
	(setf ccl::*is-logging-enabled* t)
\end{lstlisting}

The only things left to do start the logging before the plan execution and after the execution to finishing it.
It can look like the following:
\begin{lstlisting}[language=lisp, caption=Steps to Record an Episode for a \cram Plan]
	(ccl::start-episode)
	(urdf-proj:with-simulated-robot (demo::demo-random nil ))
	(ccl::stop-episode)
\end{lstlisting}
	
The generate \neem will be stored per default in "\raisebox{-0.9ex}{\~{}}/knowrob-memory".
Keep in mind to  have \knowrob launched before starting the \neem recording. 
You can start \knowrob via:

\begin{lstlisting}[language=bash, caption=How to Start \knowrob]
	roslaunch knowrob_memory knowrob.launch
\end{lstlisting}

\subsection{Data}
After you generate your first \neem~you will find a folder with a timestamp store in the "~/knowrob-memory" folder (per default).
This folder contains the \neemnar and \neemexp . The \neemnar is represented in the "beliefstate.owl". The triple store and the \neemexp are stored in the "roslog" folder. The triple store is stored in the "triples.bson". The other bsons files are presentening the logged rostopics. More about the rostopics is stated in Section \todo{put reference to rostopic section}

\subsection{Log own designed plans}
The disadvantage of having a strong semantic knowledge representation is that our ontology.
Currently, we focused on the support on setting-up and cleaning-up a table.
If you want for instance create \neems for an autonomous car, you will need to extend the \ease ontologies and the logger with your required actions, parameters etc.
In the following subsection, we will describe how you can add the required stuff so they are semantical log.
In general, please feel free to share your changes with us in form of an pull request to our repositories.
So we can provide you feedback and your help us to extend the features

\subsubsection{Adding New Tasks}
The most obvious requirement is to define your tasks.
A task might be something like cutting, stopping or accelerating.
To be able to semantaclly log the task, you will need first define the task in the \easeAct.
Make sure that the new action is a child of the \textit{CommunicationTask}, \textit{MentalTask} or \textit{PhysicalTask}.
If you will try to log unknown task, there will be logged as \textit{PhysicalTask}.
The \textit{PlanExecution} instance pointing to that \textit{PhysicalTask} , will have a comment attached with the statement "Unknown Action: <CRAM-ACTION-NAME>".
After you add the new action to the ontology, please open the "knowrob-action-name-handler.lisp" in the cram-cloud-logger package and add your new action in the format "(CRAM-ACTION-NAME EASE-ONTOLOGY-NAME)".
If this step you added successfully the support of the new action to the logger.

\subsubsection{Adding New Objects}
Unknown object will be logged as \textit{DesignedArtifact} with the comment attached "Unknown Object: CRAM-OBJECT-TYPE"
To add your object to the ontology, you need to add it in the \easeObj.
Afterwards, open the "utils-for-perform.lisp" in the 
cram-cloud-logger package and include the new object in the hash table generate in "get-ease-object-lookup-table" where the key is the CRAM-OBJECT-TYPE and the value is the uri to the object concept created in \easeObj.

\subsubsection{Adding New Failure}
Unknown failures will be logged as \textit{Failure} with the comment attached "Unknown failure: CRAM-FAILURE-NAME".
To integrate your new \cram failures in to the ontology, you need to add your new failures into the \cramOwl.
Afterwards, open "failure-handler.lisp" in the cram-cloud-logger package and your new action in the format "(CRAM-FAILURE-NAME CRAM-ONTOLOGY-NAME)".	

\subsubsection{Adding New Rostopic}
Per default, we log the rostopics \tf and tf\_static.
If you need to log additional topics, open "memory.pl" in the "knowrob\_memory" package from \knowrob and include your topic in the "mem\_episode\_start(Episode)" function.
After the \neem generation, the data will be stored in the created \neem folder under the file "roslog/<rostopic>.bson".

\subsubsection{Adding New Parameters}
Unknown parameters will be logged as comment attached to the corresponding \textit{PlanExecution} instance.
The comment statement "Unknown Parameter: PARAMETER-NAME  -\#\#\#\#- PARAMETER-VALUE/>"
The current parameter types are represented 

\todo{@Ontology group: Please make sure that is available in the ontology }

\begin{enumerate} 
	\item Integer/Floats
	\item Posen
	\item Spatial Relations
	\item Link to entities of other ontologies such as http://knowrob.org/kb/PR2.owl\#pr2\_right\_arm
\end{enumerate}

Before you want to model your parameter what data type your parameter is.
If it is a complex object, you need to consider how you want to to represent it an the ontology.
For simpler representaion such has a discrete domain representation, you might represented as the domain values as \textit{Region} and add the model the parameter as a subconcept of \textit{Parameter}.
More information about the concepts \textit{Region} and \textit{Parameter} can be found in \todo{ reference to Region and parameter}.


\subsection{Adding New Reasoning Tasks}
\todo{@Ontology group: How to log the result of the reasoning query ?}



\section{Next steps}
After you have generate your \neem, you can use the tool \todo{Add neem2narrative} to generate an cvs file for your \neem.
Keep in mind that the csv is a abstraction of \neemnar and can be used to make data-mining on explicit knowledge.
For more sophisticated analysis, you will need to use \knowrob. 
We use this general analysis to identify bottlenecks in our plan execution.
We also showed that with a collection of \neems we are able to improve the robot's performance.
\todo{Af}
The tools for the feature extraction can be found here \todo{Add link}
Now that you we encourage you to generate your \neems and share them via our \neemhub.

\subsubsection{Example}
	\label{ch:example}
	\todo{Add belief state example}
	\todo{What about reference to semantic map ?}
	In this chapter we will show a generated NEEM based on the version \neemversion.
	You can download the log used in this example here.	\todo{Provide link for downloading real log file}
	We will not go into the details about the \neemexp meaning the \tf since we just stored the exact \tf message in the database.
	Therefore it is not required to have detail explanation since the idea of \tf is already understood.
	
	This log file represents a experiment where the \pr tired to grasp five objects which were located on a kitchen counter and bring each object to our table and place them there.
	We defined grapsing + placing as a transporting.
	This experiment was performed in projection meaning in simulation.
	\cram is not used using a belief state during execution in projection, therefore we will not show how a belief state is represented.
	The object which were precived or grapsed are represent as strings.
	
	Figure \ref{fig:actionExample} show how a transporting task for a cup is represented.
	To summarize what happend during the transporitng task in the experiment,
	The task was performed not successfully in projection. 
	The failure was that the cup was unfetchable for the \pr.
	The \pr used the left arm to grasp the cup from a connected space region and transported to object to a specific pose in the map.
	\todo{How we can differ between those two goal Locations ?}
	This task has also one sub action. 
	Given the task the agent inferred that to achieve this task, it has to perform a PickingUpAnObject task.
	You can also read from the log that the \pr2 tried already an transporting task on another object and that this task is also not the last transporitng task.
	
	\begin{minipage}{\textwidth}
		\scriptsize
		\begin{lstlisting}[frame=single]
		Individual: TRANSPORTING_KVRJlsOG
		Facts: 
		goalLocation knowrob;ConnectedSpaceRegion_SgfniCWn
		goalLocation knowrob;Pose_eLrKlpcE
		arm pr2;pr2_left_arm
		endTime knowrob;timepoint_1526640320.499945
		failure CRAM-COMMON-FAILURES:OBJECT-UNFETCHABLE
		nextAction knowrob;TRANSPORTING_ysHEUIwW
		objectType "CUP"\todo{add figure to connected space region}
		performedInProjection true
		previousAction knowrob;TRANSPORTING_FkRlsFGa
		startTime knowrob;timepoint_1526640316.535484
		subAction knowrob;PickingUpAnObject_iLTYaxBE
		taskContext TableSetting
		taskSuccess false
		\end{lstlisting}

		\captionof{figure}{Example for a Transporting Task}
		\label{fig:actionExample}
	\end{minipage}
			\vspace{0.5mm}	
	
	The next transporting represented in \dots \ref{fig:actionExample2} represents a transporting task of a bowl object.
	In contrast to the previous task, this task was succuessfully performed.
	Again the \pr used only the left arm.
	The \pr grasped the object from a connected space region which is an kitchen counter \ref{fig:connectedSpaceRegion}.
	It placed the object in the specific pose in the map which is displayed in \ref{fig:pose}.
	

\begin{minipage}{\textwidth}
\scriptsize
\begin{lstlisting}[frame=single]
  Individual: TRANSPORTING_ysHEUIwW
	Facts: 
	  goalLocation knowrob;ConnectedSpaceRegion_YOXaigqs
	  goalLocation knowrob;Pose_GIcRIGdP
	  subAction knowrob;PickingUpAnObject_CTGdupxa
      subAction knowrob;PuttingDownAnObject_uBRtvLcg
      arm pr2;pr2_right_arm
      endTime knowrob;timepoint_1526640332.733415
      nextAction knowrob;TRANSPORTING_HRgQoeEw
      objectType rdf:resource="BOWL"
      performedInProjection true
      previousAction knowrob;TRANSPORTING_KVRJlsOG
      startTime rdf:resource="&knowrob;timepoint_1526640320.561752
	  taskContext "TableSetting"
	  taskSuccess true
\end{lstlisting}
\vspace{2mm}
\captionof{figure}{Example for a second Transporting Task}
\label{fig:actionExample2}
\end{minipage}


\begin{minipage}{\textwidth}
	\scriptsize
\begin{lstlisting}[frame=single]
  Individual: ConnectedSpaceRegion_YOXaigqs
	Facts: 
	  onPhysical knowrob:iai_kitchen_sink_area_counter_top
\end{lstlisting}
	\vspace{2mm}
	\captionof{figure}{Example for Connected Space Region}
	\label{fig:connectedSpaceRegion}
\end{minipage}


\begin{minipage}{\textwidth}
	\scriptsize
\begin{lstlisting}[frame=single]
  Individual: Pose_GIcRIGdP
    Facts: 
	  quaternion 0.0 0.0 1.0 0.0
	  translation -0.7599999904632568 1.190000057220459 0.9300000071525574
\end{lstlisting}
	\vspace{2mm}
	\captionof{figure}{Example for a Pose}
	\label{fig:pose}
\end{minipage}


\section{VR \neems}
\label{sec:vr-neem}
\lstset{style=lispcode}

This section will describe how \neems can be generated within a Virtual Reality environment and how they can be utilized within \cram plans to help a robot perform everyday household activities. The use of VR allows us as humans to show the robot an action we want it to perform within a variety of different environments. This facilitates learning of  a lot of common sense knowledge, e.g. where the objects necessary to perform a specific action are commonly stored within the environment, which objects are needed for a specific action, where the human user was standing when he was grasping a certain object, how the objects were arranged on a surface relative to one another and how the human user grasped them.  

The example scenario used here is the breakfast setting scenario. This means that the robot is supposed to set up the table with a bowl, cup and a spoon in preparation of a breakfast cereal meal. \todo{@Alina: move this probably to a different section.}
% Maybe make a section describing all the advantages it brings to use VR? But maybe this is also enough

\subsection{Prerequisite}
%Everythhing needed to be able to record NEEMs. Unreal, Plugins, Kitchen, RobCog, robcog_knowrob (for quering). Refer to RobCog and some of the CRAM-VR Tutorials. Also setup kitchen with all the arrays/Links.
Before VR-\neem generation can take place, the proper VR-environment needs to be set up within the Unreal Engine, including the installation of the USemLog Plugin, which records the \neems and generates the appropriate .owl files. The plugin and a setup of a kitchen environment can be found within the RobCog project. 
The KnowRob and MongoDB installation are the same as in the section above. \todo{@Alina add links to the sections}
In order to be able to use the \neems within \cram, the data first needs to be transferred into the MongoDB and KnowRob. This can be achieved by running the \textit{vr neems to knowrob} scripts. Please refer to the README.md for execution examples.

 
To summarize: 
\begin{itemize}
	\item Unreal Engine\footnote{\url{https://www.unrealengine.com/}} Version 4.22.3
	\item RobCog\footnote{\url{https://github.com/robcog-iai/RobCoG}}
	\item vr\_neems\_to\_knowrob\footnote{\url{https://github.com/ease-crc/vr_neems_to_knowrob}} 
	\item knowrob\_robcog \todo{@Alina add link}
	\item CRAM\footnote{\url{https://github.com/cram2/cram.git}} Branch: boxy-melodic

\end{itemize}

\subsection{Recording Virtual Reality Narrative Enabled Episodic Memories}
%Run around in VR and do stuff, dump semantic map using plug in
\todo{@Alina add screenshots}
Check if all items that you want to appear in the \neems, have a tag under which they will be represented within the ontology. You can check the tag by clicking any item within the kitchen environment, and going to the \textit{Actor} section in the \textit{details} pane. Click the little arrow to expand the section, and also expand the \textit{Tags} section. There you should see something like this:

\begin{lstlisting}

SemLog;Class,IAIIslandArea;Id,tpzV6l885UGL785BwZFHYQ;

\end{lstlisting}
This string defines the class of the item and it's id. More information can be added here, depending on the item and the ontology and the level of detail desired within the classification. This is also very important when introducing new items to the Virtual Reality setup. Items which do not have these tags, will not be included in the recorded \neem data. 

Once everything is set up and all potentially new items are within the virtual reality environment, the recording can begin. First, the semantic map can be automatically generated by going to the \textit{RobCog} pane within the \textit{Unreal Engine} editor, and clicking the \textit{SemanticMap} button. 
\todo{@Alina add Screenshot}
After this, the generated semantic map should be located in the \textit{RobCog/Episodes} directory. It contains the initial state of the virtual reality environment, including the position and rotation of all the furniture objects and also their classifications. 

The next step is then to start the virtual reality simulation and perform some actions within it, e.g. setting up a breakfast table. The position of the VR-headset and controllers is being tracked the entire time, as well as the interactions of the virtual hands with the environment and objects. Picking up an object would generate a \textit{GraspingSomething}-action. Placing an object down on a table would generate a \textit{Contact}-Action between the object and the surface it has been placed upon. All these interactions can later on be queried for. 

Once all the desired actions are complete, the simulation can be stopped and an \textit{EventData\_ID} directory appears in the \textit{RobCog/Episodes} directory. It contains and \textit{EventData\_ID.owl} file, an \textit{EventData\_ID.html} file, which visualizes all the occurred events, and a \textit{RawData\_ID.json}, which contains all the information about the performed actions and events. The last file is the one that needs to be uploaded into the \textit{MongoDB}. 


\subsection{Transferring VR-NEEMs into the Knowledge Base}
%Use the scripts. Maybe put the scripts into a different repo. Maybe RobCog? -> make own knowrob_robcog fork and put it there.
Please refer to the README of these scripts \url{https://github.com/ease-crc/vr_neems_to_knowrob} in order to import the VR-\neems into KnowRob and MongoDB. More information about the import, how it generally works and why the scripts were created the way they are, please refer to: \url{http://cram-system.org/tutorials/advanced/unreal#importing_new_episode_data_into_mongodb_and_knowrob_additional_information}. 

\subsection{Using VR-NEEM Data in CRAM plans}
%Which data is used, how is it used/sampled. Add reference to Paper maybe
There is a demo within \cram which uses the data collected in VR, including a tutorial on how to run it. It can be found here: \url{http://cram-system.org/tutorials/advanced/unreal}. In this demo the robot performs a pick and place task, picking up a cup, bowl and a spoon from the sink counter, and placing them onto the kitchen island. In order to do this, \cram queries KnowRob for the following information: 

\begin{itemize}
	\item where/from which surface was the object picked up?
	\item where was the human user standing when he was picking up/placing the object?
	\item on which surface and where was the object placed? (In relation to other objects)
	\item with which hand was the object grasped?
	\item from which direction (top,left,right...) was the object grasped?
\end{itemize}

Since the virtual reality kitchen can look very different than the one the robot is acting in, all of the poses are calculated relative to the respective surfaces and each other. For example, the spoon is always placed to the right of the bowl. 

For more information on how \cram interacts with KnowRob and how json-prolog can be used within \cram, please refer to \url{http://cram-system.org/tutorials/intermediate/json_prolog}




\subsection{Future Work}
%references to future work, aka. my masters thesis and ML to showcase how else VR-NEEms could be used.



\chapter{NEEM-Hub}
\label{ch:neemhub}

\section{Publishing}
In this chapter we have to describe, how the NEEM narrative and NEEM experience should be represented e.g.\ using OWL and MongoDB and how the partners can upload those file to openEASE.

\section{Maintaining}

\chapter{Application}
\section{Question Answering}
\section{Machine Learning}

\chapter{Future Work}

\section{NEEM-Experience}
\begin{enumerate}
	\item Logging Images
	\item CostMaps
\end{enumerate}

\section{NEEM-Narrative}
\begin{enumerate}
	\item Logging Reasoning Tasks
\end{enumerate}

\bibliography{neem}{}
\bibliographystyle{plain}
\end{document}

\end{document}
