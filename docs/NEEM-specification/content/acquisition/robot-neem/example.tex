\subsubsection{Example}
	\label{ch:example}
	\todo{Add belief state example}
	\todo{What about reference to semantic map ?}
	In this chapter we will show a generated NEEM based on the version \neemversion.
	You can download the log used in this example here.	\todo{Provide link for downloading real log file}
	We will not go into the details about the \neemexp meaning the \tf since we just stored the exact \tf message in the database.
	Therefore it is not required to have detail explanation since the idea of \tf is already understood.
	
	This log file represents a experiment where the \pr tired to grasp five objects which were located on a kitchen counter and bring each object to our table and place them there.
	We defined grapsing + placing as a transporting.
	This experiment was performed in projection meaning in simulation.
	\cram is not used using a belief state during execution in projection, therefore we will not show how a belief state is represented.
	The object which were precived or grapsed are represent as strings.
	
	Figure \ref{fig:actionExample} show how a transporting task for a cup is represented.
	To summarize what happend during the transporitng task in the experiment,
	The task was performed not successfully in projection. 
	The failure was that the cup was unfetchable for the \pr.
	The \pr used the left arm to grasp the cup from a connected space region and transported to object to a specific pose in the map.
	\todo{How we can differ between those two goal Locations ?}
	This task has also one sub action. 
	Given the task the agent inferred that to achieve this task, it has to perform a PickingUpAnObject task.
	You can also read from the log that the \pr2 tried already an transporting task on another object and that this task is also not the last transporitng task.
	
	\begin{minipage}{\textwidth}
		\scriptsize
		\begin{lstlisting}[frame=single]
		Individual: TRANSPORTING_KVRJlsOG
		Facts: 
		goalLocation knowrob;ConnectedSpaceRegion_SgfniCWn
		goalLocation knowrob;Pose_eLrKlpcE
		arm pr2;pr2_left_arm
		endTime knowrob;timepoint_1526640320.499945
		failure CRAM-COMMON-FAILURES:OBJECT-UNFETCHABLE
		nextAction knowrob;TRANSPORTING_ysHEUIwW
		objectType "CUP"\todo{add figure to connected space region}
		performedInProjection true
		previousAction knowrob;TRANSPORTING_FkRlsFGa
		startTime knowrob;timepoint_1526640316.535484
		subAction knowrob;PickingUpAnObject_iLTYaxBE
		taskContext TableSetting
		taskSuccess false
		\end{lstlisting}

		\captionof{figure}{Example for a Transporting Task}
		\label{fig:actionExample}
	\end{minipage}
			\vspace{0.5mm}	
	
	The next transporting represented in \dots \ref{fig:actionExample2} represents a transporting task of a bowl object.
	In contrast to the previous task, this task was succuessfully performed.
	Again the \pr used only the left arm.
	The \pr grasped the object from a connected space region which is an kitchen counter \ref{fig:connectedSpaceRegion}.
	It placed the object in the specific pose in the map which is displayed in \ref{fig:pose}.
	

\begin{minipage}{\textwidth}
\scriptsize
\begin{lstlisting}[frame=single]
  Individual: TRANSPORTING_ysHEUIwW
	Facts: 
	  goalLocation knowrob;ConnectedSpaceRegion_YOXaigqs
	  goalLocation knowrob;Pose_GIcRIGdP
	  subAction knowrob;PickingUpAnObject_CTGdupxa
      subAction knowrob;PuttingDownAnObject_uBRtvLcg
      arm pr2;pr2_right_arm
      endTime knowrob;timepoint_1526640332.733415
      nextAction knowrob;TRANSPORTING_HRgQoeEw
      objectType rdf:resource="BOWL"
      performedInProjection true
      previousAction knowrob;TRANSPORTING_KVRJlsOG
      startTime rdf:resource="&knowrob;timepoint_1526640320.561752
	  taskContext "TableSetting"
	  taskSuccess true
\end{lstlisting}
\vspace{2mm}
\captionof{figure}{Example for a second Transporting Task}
\label{fig:actionExample2}
\end{minipage}


\begin{minipage}{\textwidth}
	\scriptsize
\begin{lstlisting}[frame=single]
  Individual: ConnectedSpaceRegion_YOXaigqs
	Facts: 
	  onPhysical knowrob:iai_kitchen_sink_area_counter_top
\end{lstlisting}
	\vspace{2mm}
	\captionof{figure}{Example for Connected Space Region}
	\label{fig:connectedSpaceRegion}
\end{minipage}


\begin{minipage}{\textwidth}
	\scriptsize
\begin{lstlisting}[frame=single]
  Individual: Pose_GIcRIGdP
    Facts: 
	  quaternion 0.0 0.0 1.0 0.0
	  translation -0.7599999904632568 1.190000057220459 0.9300000071525574
\end{lstlisting}
	\vspace{2mm}
	\captionof{figure}{Example for a Pose}
	\label{fig:pose}
\end{minipage}
