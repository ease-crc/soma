
% % % % % % % % % % % % % % % % % % % % % % % %
% % % Prelude
% % % % % % % % % % % % % % % % % % % % % % % %
\newcommand{\givenODPNAME}{}
\newcommand{\givenODPINTENT}{}
\newcommand{\givenODPDEFINEDIN}{}
\newcommand{\givenODPDESCRIPTION}{}
\newcommand{\givenODPGRAPHIC}{}
\newcommand{\givenODPDOMAIN}{}
\newcommand{\givenODPQUESTION}{}
\newcommand{\ODPINTENT}[1]     {\renewcommand{\givenODPINTENT}{#1}}
\newcommand{\ODPDEFINEDIN}[1]  {\renewcommand{\givenODPDEFINEDIN}{#1}}
\newcommand{\ODPDESCRIPTION}[1]{\renewcommand{\givenODPDESCRIPTION}{#1}}
\newcommand{\ODPGRAPHIC}[1]    {\renewcommand{\givenODPGRAPHIC}{#1}}
\newcommand{\ODPDOMAIN}[1]     {\renewcommand{\givenODPDOMAIN}{#1}}
\newcommand{\ODPQUESTION}[1]   {\renewcommand{\givenODPQUESTION}{#1}}
\newcommand{\OPDinit}{
  \renewcommand{\givenODPINTENT}{REQUIRED!}
  \renewcommand{\givenODPDEFINEDIN}{REQUIRED!}
  \renewcommand{\givenODPDESCRIPTION}{REQUIRED!}
  \renewcommand{\givenODPGRAPHIC}{REQUIRED!}
  \renewcommand{\givenODPQUESTION}{}
  \renewcommand{\givenODPDOMAIN}{}
  \renewcommand{\labelitemi}{$\mathbf{\sqsubseteq}$}
}

\newenvironment{owlclass}[2][,] {
  \begin{minipage}{4.0cm}
  \begin{center}
  \texttt{\bf#2} \\[-0.2cm]
  \par\noindent\rule{\textwidth}{0.4pt}
  \vspace{-0.6cm}
  \begin{itemize}[#1]
  \raggedright} {
  % % % % % %
  \end{itemize}
  \end{center}
  \end{minipage}
}

\newenvironment{ODP}[1]{
\OPDinit
\renewcommand{\givenODPNAME}{#1}
}{
\givenODPDESCRIPTION
\begin{figure}[htb]
\begin{minipage}{0.45\textwidth}
\begin{tabular}{ p{1.8cm} p{3.2cm} }
\toprule
% {\it\bf Name}                 & \emph{\givenODPNAME} \\
{\it\bf Intent}               & \givenODPINTENT \\
{\it\bf Domains}              & \givenODPDOMAIN \\
{\it\bf Competency Questions} & \givenODPQUESTION \\
{\it\bf Defined in}           & \givenODPDEFINEDIN \\
\bottomrule
\end{tabular}
\end{minipage}
\begin{minipage}{0.55\textwidth}
\begin{center}
\givenODPGRAPHIC
\end{center}
\end{minipage}
\caption{The \emph{\givenODPNAME} ODP.}
\end{figure}
}

\tikzset{owlclass/.style={draw=blue!40,fill=blue!20,rounded corners}}

% % % % % % % % % % % % % % % % % % % % % % % %
% % % % % % % % % % % % % % % % % % % % % % % %
% % % % % % % % % % % % % % % % % % % % % % % %
\chapter{Ontology Design Patterns (ODPs)}

% % % % % % % % % % % % % % % % % % % % % % % %
% \section{Object Roles}

% % % % % % % % % % % % % % % % % % % % % % % %
% % % % % % % % % % % % % % % % % % % % % % % %
% \section{Quality and Quality Regions}

% % % % % % % % % % % % % % % % % % % % % % % %
% % % % % % % % % % % % % % % % % % % % % % % %
% \section{Designed Artifacts}

% % % % % % % % % % % % % % % % % % % % % % % %
% % % % % % % % % % % % % % % % % % % % % % % %
\section{Task Execution ODP}
\begin{ODP}{Task Execution}
\ODPDESCRIPTION{
This ODP allows to make assertions on roles played by agents
without involving the agents that play that roles, and vice versa.
It allows to express neither the context type in which tasks are defined,
not the particular context in which the action is carried out.
Moreover, it does not allow to express the time at which
the task is executed through the action
(for actions that do not solely execute that certain task).}
\ODPINTENT{To represent actions through which tasks are executed. }
\ODPDOMAIN{
  \texttt{Organization},
  \texttt{Management},
  \texttt{Scheduling},
  \texttt{Workflow}}
\ODPDEFINEDIN{DUL.owl}
\ODPQUESTION{
  \emph{Which task is executed through this action?}
  \emph{What actions can execute that task?}}
\ODPGRAPHIC{
\begin{tikzpicture}
 \node[owlclass] (ACTION) {
 \begin{owlclass}{Action}
  \item $(\geq 1\ \emph{executesTask}.\texttt{Task})$
 \end{owlclass}
 };
 \node[owlclass,below=0.6cm of ACTION] (TASK) {
 \begin{owlclass}{Task}
  \item $(\forall \emph{isExecutedIn}.\texttt{Action})$
 \end{owlclass}
 };
 \draw (ACTION) edge[thick,-,dashed,blue!60] (TASK);
\end{tikzpicture}
}
\end{ODP}

% % % % % % % % % % % % % % % % % % % % % % % %
% % % % % % % % % % % % % % % % % % % % % % % %
% \section{Workflows}

% % % % % % % % % % % % % % % % % % % % % % % %
% % % % % % % % % % % % % % % % % % % % % % % %
% \section{Action Phases}

% % % % % % % % % % % % % % % % % % % % % % % %
% % % % % % % % % % % % % % % % % % % % % % % %
% \section{Transients}

