
%%%%%%%%%%%%%%%%%%%%%%%%%%%%%%%%%%%%%%%%%%%%%%%
%%%%%%%%%%%%%%%%%%%%%%%%%%%%%%%%%%%%%%%%%%%%%%%
\section{NEEM-Background}
\label{ch:background}

%%%%%%%%%%%%%%%%%%%%%%%%%%%%%%%%%%%%%%%%%%%%%%%
%%%%%%%%%%%%%%%%%%%%%%%%%%%%%%%%%%%%%%%%%%%%%%%
\subsection{Taxonomy}
\subsubsection{Physical Objects}
According to \dul~\footnote{\url{http://www.ontologydesignpatterns.org/ont/dul/DUL.owl}} upper level ontology, an object participates in an event during its lifetime and has its own spatial location.  describes objects into several subcategories which includes agent, digital object, feature, physical object, social object and transient. \todo{Seba: Is something missing here ?}


%An object branch also covers a design taxonomy which considers functional, structural and aesthetic aspect of object design. Designs are useful to support an agent to hypothesize unknown functions that can be served by an entity\todo{Seba; What exactly is en entity ? An object ? an instance of a concept ?}. A design describes objects which host a common design relevant qualities such as, dispositional, geometrical, and aesthetic. An intelligent agent would be able to infer based on dispositional quality of an object if it can be used to serve other function in everyday task. For example, a heavy door stopper would also be able to function as paper weight or a dining table can be also used as ping pong table based on appropriate dimensions.
\todo{Seba: Will the qualities be somewhere defined ? Answer: in neem background}
\subsubsection{Qualities}

%%%%%%%%%%%%%%%%%%%%%%%%%%%%%%%%%%%%%%%%%%%%%%%
%%%%%%%%%%%%%%%%%%%%%%%%%%%%%%%%%%%%%%%%%%%%%%%
\subsection{Relationships}
%%%%%%%%%%%%%%%%%%%%%%%%%%%%%%%%%%%%%%%%%%%%%%%
\subsubsection{Componency}

%%%%%%%%%%%%%%%%%%%%%%%%%%%%%%%%%%%%%%%%%%%%%%%
\subsubsection{Properties}
\label{sec:qualification}

Qualities are the properties of an object that are not part of it, but cannot exist without it.
This is, for example, the quality of having a shape -- a quality inherited by all physical objects.
Another example is the quality of a floor being slippery.
A robot navigating on such a floor could use this knowledge to avoid, for example,
spillage when moving on the floor with a coffee-filled cup.
The quality concept does not directly encode the value of the object property, but only focusses on characteristics of the property itself.
This is mainly useful in cases where individual aspects of an entity are considered in the domain of discourse.

\begin{ODP}{Object Qualities}
	\ODPINTENT{To represent the qualities of an object.}
	\ODPDEFINEDIN{DUL.owl}
	\ODPQUESTION{What qualities does this object have? That are the objects with this quality?}
	\ODPGRAPHIC{
	\begin{tikzpicture}
	    \node[owlclass] (A) {Object};
	    \node[owlclass,below=0.6cm of A] (B) {Quality};
	    \draw (A) edge[relation] node[midway,label=right:hasQuality] {} (B);
 %\node[owlclass] (QUALITY) {
 %\begin{owlclass}{Quality}
 % \item $(\exists \emph{isQualityOf}.\texttt{Entity})$
 % \item $(\exists \emph{hasRegion}.\texttt{Region})$
 %\end{owlclass}
 %};
 %\node[owlclass,below=0.6cm of QUALITY] (REGION) {
 %\begin{owlclass}{Region}
 % \item $(\exists \emph{isRegionFor}.\texttt{Quality})$
 %\end{owlclass}
 %};
 %\draw (QUALITY) edge[thick,-,dashed,blue!60] (REGION);
	\end{tikzpicture}
	}
	%% Example KnowRob language expressions
	\ODPEXAMPLES{
		\emph{has\_quality($x$,$y$)} &
		$y$ is a quality of $x$
	}
\end{ODP}

Each object property has one value at a time. The value of an object property is called \emph{region}.
The value itself is an element, or a sub-region in some dimensional
space such as \emph{time interval} or \emph{space region}.
A region may be a finite set of discrete labels, allowing for ``qualitative'' descriptions,
but more often a region is some dimensional space allowing ``quantitative'' descriptions.
A Region may contain a single point, in cases where the value of a property is known precisely.
Note that the domain of the relation \emph{hasRegion} is not \emph{Quality} but \emph{Entity}.
This is to allow assigning regions to entities without explicating the quality as a concept (quality-as-relation).
\todo{Seba: maybe some more example would be nice. Or refering to the example with the slippery floor. }

\begin{ODP}{Regions}
	\ODPINTENT{To represent values of attributes of things.}
	\ODPDEFINEDIN{DUL.owl}
	\ODPQUESTION{What is the value for the attribute of that entity? Which entities have a certain value on that parameter/attribute/feature?}
	\ODPGRAPHIC{
	\begin{tikzpicture}
	    \node[owlclass] (A) {Entity};
	    \node[owlclass,below=0.6cm of A] (B) {Region};
	    \node[data,below=0.6cm of B] (C) {XSD Type};
	    \draw (A) edge[relation] node[midway,label=right:hasRegion] {} (B);
	    \draw (B) edge[relation] node[midway,label=right:hasRegionDataValue] {} (C);
	\end{tikzpicture}
	}
	%% Example KnowRob language expressions
	\ODPEXAMPLES{
		\emph{has\_region($x$,$y$)} &
		$y$ is a region of $x$ \\
		% % % % %
		\emph{has\_data\_value($x$,$y$)} &
		$y$ is a data value of $x$ 
	}
\end{ODP}

%%%%%%%%%%%%%%%%%%%%%%%%%%%%%%%%%%%%%%%%%%%%%%%
%%%%%%%%%%%%%%%%%%%%%%%%%%%%%%%%%%%%%%%%%%%%%%%
\subsection{Format}
\subsubsection{URDF}

