\chapter{NEEM-Acquisition}
\label{ch:acquisition}
\chapterauthor{S. Koralewski, A. Hawkin}

This chapter focuses on the acquisition process of \neems.
At first, we will provide the tools and procedures to acquire episodic memories from robots performing experiments.
The second section focuses on the \neem acquisition from virtual reality. 

%Each section will contain an example \neem to provide insights on, how the representation, described in the chapters \ref{ch:background}, \ref{ch:narrative} and \ref{ch:experience} is utilized to capture performed activities by robots or by humans. 
%In addition, each example \neem is available on the \neemhub for downloading.



\section{Data Structure}

We are using MongoDB to capture the data structures of the \neems.
If you will use the \knowrob interface to create your \neems then your \neem will consist of at least 3 folders - \textit{annotations}, \textit{inferred} and \textit{triples}.
The \neemnar and \neemexp are stored as a collection of BSON \footnote{http://bsonspec.org/faq.html} files.
Each folder should contain a BSON file and metafile stored as JSON. The metafile will include additional information related to \neems. This additional meta information is useful for searching \neem on \openease platform and hence needs to be provided by \neem creator while \neem acquisition time. An example of such information is as displayed below:

\begin{lstlisting}[language=json,firstnumber=1]
{
	"_id" : ObjectId("5f22b1f512db5aed7cd1961b"), 
	"created_by" : "seba",
	"created_at" : "2020-07-21T06:54:25+00:00",
	"model_version" : "0.1",
	"description" : "NEEM for robot making pizza.",
	"keywords" : [	
	"Pizza",
	"Robot"
	],
	"url" : "Placeholder for the NEEM hub repository url",
	"name" : "NEEM for robot making pizza",
	"activity" : {
		"name" : "Pizza making",
		"url" : "Placeholder for the url/uri of Activity concept defined in ontology"    
	},
	"environment" : "Kitchen",
	"image" : "placeholder for image url for showing neem image on openEASE",  
	"agent" : "Robot"
}
	
\end{lstlisting}

Each generated \neem stores also the complete state of the \soma ontology which was used during the acquisition process.
The benefit of this is that while loading a \neem, it is not required to keep track to load the correct \soma version.
In the following, we will give an overview which information is contained in those folders generated by \knowrob:


\begin{description}
	\item[\textbf{annotations}] The annotations collection contains annotations(comments) which are asserted to the concepts of the ontology.
	\item[\textbf{inferred}] The inferred collection contains triples which were inferred and not asserted during the logging process. Inference processes can be triggered when triples are asserted directly to the knowledge base.
	\item[\textbf{triples}] The triples collection contains all triples which were asserted into the knowledge base during run time.
\end{description}

\subsection{Triple data as JSON object}
	Triple data can also be provided in form of JSON documents, where triples are represented as subject, predicate and object. Subjects and objects are identified by an Internationalized Resource Identifier (IRI), which is pointing to concepts or instances defined in the \soma ontology. A triple can either link subject with an object, or can link subject with data value which is represented using one of the base types: string, boolean, and a number. Whereas, the predicate is named by the IRI pointing to property concepts mentioned in the \soma ontology. It is also possible to provide additional time scope fields `since` and `until` to indicate that the given triple is valid for the given time scope. These values are considered here in seconds from when an experiment has started being recorded. An example of such a JSON document is given below where the \emph{Salad\_PMRVYPJH} has a \emph{patient} role from 27.739th second till 29.075. By default triple is valid for the infinite time when scope parameters are not specified. 
\begin{lstlisting}[language=json,firstnumber=1]
[
 {
   "s": "http://www.ease-crc.org/ont/SOMA.owl#Salad_PMRVYPJH",
   "p": "http://www.ontologydesignpatterns.org/ont/dul/DUL.owl#hasRole",
   "o": "http://www.ease-crc.org/ont/SOMA.owl#Patient_YGJUVNDR",
   "since": 27.739,
   "until": 29.075
 }
]
\end{lstlisting}

It is important to note that, this is an intermediate data format which is not equivalent with how the NEEM narrative is actually stored in databases.
The format described here rather serves as an easy-to-use interchange format.

\section{Narrative Enabled Episodic Memories for Robotic Agents}
\label{sec:robot-neem}
\lstset{style=lispcode}
This section focuses on describing how to generate \neems from experiments performed by the robot. 

\subsection{Prerequisite}

Before you are intending to generate your episodic memories, make sure you are familiar with the Cognitive Robot Abstract Machine (\cram)\footnote{\url{http://cram-system.org/cram}} system and installed it on your machine.
\cram is a cognitive-enabled planning framework which allows to design high-level plan for robots.\todo{@Seba: Add more detailed description}
This section requires that your robot plan is written in \cram to be able to generate \neems.
However, once you are familiar with our planning and logging components, you will be able to port those components to your preferred planning tool.

In addition to having a valid \cram plan, you will need the following software components to be installed:
\begin{itemize}
	\item A MongoDB server with at least version 3.4.10\footnote{\url{https://www.mongodb.com/}}
	\item \knowrob\footnote{\url{https://github.com/knowrob/knowrob}}
	\item \soma ontology\footnote{\url{https://github.com/ease-crc/soma}}
	\item \cram ontology\footnote{\url{https://github.com/ease-crc/cram\_knowledge}}
\end{itemize}


\subsection{Recording Narrative Enabled Episodic Memories}
Our recording mechanism  captures every executed \cram action and its parameter.
In addition, the logger puts the actions in relation to each other by creating a hierarchy which is described in Section \ref{ch:narrative,sec:actionHierarchy}.

Before you can begin to record your own \neems, you need to include the "cram-cloud-logger" package into your \cram plan.
After you included the package, you need to enable the logging via:


\begin{lstlisting}[language=lisp, caption=Enabling \neem Logging in a \cram plan]
	(setf ccl::*is-logging-enabled* t)
\end{lstlisting}

The only things left to do start the logging before the plan execution and after the execution to finishing it.
It can look like the following:
\begin{lstlisting}[language=lisp, caption=Steps to Record an Episode for a \cram Plan]
	(ccl::start-episode)
	(urdf-proj:with-simulated-robot (demo::demo-random nil ))
	(ccl::stop-episode)
\end{lstlisting}
	
The generate \neem will be stored per default in "\raisebox{-0.9ex}{\~{}}/knowrob-memory".
Keep in mind to  have \knowrob launched before starting the \neem recording. 
You can start \knowrob via:

\begin{lstlisting}[language=bash, caption=How to Start \knowrob]
	roslaunch knowrob_memory knowrob.launch
\end{lstlisting}

\subsection{Data}
After you generate your first \neem~you will find a folder with a timestamp store in the "~/knowrob-memory" folder (per default).
This folder contains the \neemnar and \neemexp . The \neemnar is represented in the "beliefstate.owl". The triple store and the \neemexp are stored in the "roslog" folder. The triple store is stored in the "triples.bson". The other bsons files are presentening the logged rostopics. More about the rostopics is stated in Section \todo{put reference to rostopic section}

\subsection{Log own designed plans}
The disadvantage of having a strong semantic knowledge representation is that our ontology.
Currently, we focused on the support on setting-up and cleaning-up a table.
If you want for instance create \neems for an autonomous car, you will need to extend the \ease ontologies and the logger with your required actions, parameters etc.
In the following subsection, we will describe how you can add the required stuff so they are semantical log.
In general, please feel free to share your changes with us in form of an pull request to our repositories.
So we can provide you feedback and your help us to extend the features

\subsubsection{Adding New Tasks}
The most obvious requirement is to define your tasks.
A task might be something like cutting, stopping or accelerating.
To be able to semantaclly log the task, you will need first define the task in the \easeAct.
Make sure that the new action is a child of the \textit{CommunicationTask}, \textit{MentalTask} or \textit{PhysicalTask}.
If you will try to log unknown task, there will be logged as \textit{PhysicalTask}.
The \textit{PlanExecution} instance pointing to that \textit{PhysicalTask} , will have a comment attached with the statement "Unknown Action: <CRAM-ACTION-NAME>".
After you add the new action to the ontology, please open the "knowrob-action-name-handler.lisp" in the cram-cloud-logger package and add your new action in the format "(CRAM-ACTION-NAME EASE-ONTOLOGY-NAME)".
If this step you added successfully the support of the new action to the logger.

\subsubsection{Adding New Objects}
Unknown object will be logged as \textit{DesignedArtifact} with the comment attached "Unknown Object: CRAM-OBJECT-TYPE"
To add your object to the ontology, you need to add it in the \easeObj.
Afterwards, open the "utils-for-perform.lisp" in the 
cram-cloud-logger package and include the new object in the hash table generate in "get-ease-object-lookup-table" where the key is the CRAM-OBJECT-TYPE and the value is the uri to the object concept created in \easeObj.

\subsubsection{Adding New Failure}
Unknown failures will be logged as \textit{Failure} with the comment attached "Unknown failure: CRAM-FAILURE-NAME".
To integrate your new \cram failures in to the ontology, you need to add your new failures into the \cramOwl.
Afterwards, open "failure-handler.lisp" in the cram-cloud-logger package and your new action in the format "(CRAM-FAILURE-NAME CRAM-ONTOLOGY-NAME)".	

\subsubsection{Adding New Rostopic}
Per default, we log the rostopics \tf and tf\_static.
If you need to log additional topics, open "memory.pl" in the "knowrob\_memory" package from \knowrob and include your topic in the "mem\_episode\_start(Episode)" function.
After the \neem generation, the data will be stored in the created \neem folder under the file "roslog/<rostopic>.bson".

\subsubsection{Adding New Parameters}
Unknown parameters will be logged as comment attached to the corresponding \textit{PlanExecution} instance.
The comment statement "Unknown Parameter: PARAMETER-NAME  -\#\#\#\#- PARAMETER-VALUE/>"
The current parameter types are represented 

\todo{@Ontology group: Please make sure that is available in the ontology }

\begin{enumerate} 
	\item Integer/Floats
	\item Posen
	\item Spatial Relations
	\item Link to entities of other ontologies such as http://knowrob.org/kb/PR2.owl\#pr2\_right\_arm
\end{enumerate}

Before you want to model your parameter what data type your parameter is.
If it is a complex object, you need to consider how you want to to represent it an the ontology.
For simpler representaion such has a discrete domain representation, you might represented as the domain values as \textit{Region} and add the model the parameter as a subconcept of \textit{Parameter}.
More information about the concepts \textit{Region} and \textit{Parameter} can be found in \todo{ reference to Region and parameter}.


\subsection{Adding New Reasoning Tasks}
\todo{@Ontology group: How to log the result of the reasoning query ?}



\section{Next steps}
After you have generate your \neem, you can use the tool \todo{Add neem2narrative} to generate an cvs file for your \neem.
Keep in mind that the csv is a abstraction of \neemnar and can be used to make data-mining on explicit knowledge.
For more sophisticated analysis, you will need to use \knowrob. 
We use this general analysis to identify bottlenecks in our plan execution.
We also showed that with a collection of \neems we are able to improve the robot's performance.
\todo{Af}
The tools for the feature extraction can be found here \todo{Add link}
Now that you we encourage you to generate your \neems and share them via our \neemhub.

\subsubsection{Example}
	\label{ch:example}
	\todo{Add belief state example}
	\todo{What about reference to semantic map ?}
	In this chapter we will show a generated NEEM based on the version \neemversion.
	You can download the log used in this example here.	\todo{Provide link for downloading real log file}
	We will not go into the details about the \neemexp meaning the \tf since we just stored the exact \tf message in the database.
	Therefore it is not required to have detail explanation since the idea of \tf is already understood.
	
	This log file represents a experiment where the \pr tired to grasp five objects which were located on a kitchen counter and bring each object to our table and place them there.
	We defined grapsing + placing as a transporting.
	This experiment was performed in projection meaning in simulation.
	\cram is not used using a belief state during execution in projection, therefore we will not show how a belief state is represented.
	The object which were precived or grapsed are represent as strings.
	
	Figure \ref{fig:actionExample} show how a transporting task for a cup is represented.
	To summarize what happend during the transporitng task in the experiment,
	The task was performed not successfully in projection. 
	The failure was that the cup was unfetchable for the \pr.
	The \pr used the left arm to grasp the cup from a connected space region and transported to object to a specific pose in the map.
	\todo{How we can differ between those two goal Locations ?}
	This task has also one sub action. 
	Given the task the agent inferred that to achieve this task, it has to perform a PickingUpAnObject task.
	You can also read from the log that the \pr2 tried already an transporting task on another object and that this task is also not the last transporitng task.
	
	\begin{minipage}{\textwidth}
		\scriptsize
		\begin{lstlisting}[frame=single]
		Individual: TRANSPORTING_KVRJlsOG
		Facts: 
		goalLocation knowrob;ConnectedSpaceRegion_SgfniCWn
		goalLocation knowrob;Pose_eLrKlpcE
		arm pr2;pr2_left_arm
		endTime knowrob;timepoint_1526640320.499945
		failure CRAM-COMMON-FAILURES:OBJECT-UNFETCHABLE
		nextAction knowrob;TRANSPORTING_ysHEUIwW
		objectType "CUP"\todo{add figure to connected space region}
		performedInProjection true
		previousAction knowrob;TRANSPORTING_FkRlsFGa
		startTime knowrob;timepoint_1526640316.535484
		subAction knowrob;PickingUpAnObject_iLTYaxBE
		taskContext TableSetting
		taskSuccess false
		\end{lstlisting}

		\captionof{figure}{Example for a Transporting Task}
		\label{fig:actionExample}
	\end{minipage}
			\vspace{0.5mm}	
	
	The next transporting represented in \dots \ref{fig:actionExample2} represents a transporting task of a bowl object.
	In contrast to the previous task, this task was succuessfully performed.
	Again the \pr used only the left arm.
	The \pr grasped the object from a connected space region which is an kitchen counter \ref{fig:connectedSpaceRegion}.
	It placed the object in the specific pose in the map which is displayed in \ref{fig:pose}.
	

\begin{minipage}{\textwidth}
\scriptsize
\begin{lstlisting}[frame=single]
  Individual: TRANSPORTING_ysHEUIwW
	Facts: 
	  goalLocation knowrob;ConnectedSpaceRegion_YOXaigqs
	  goalLocation knowrob;Pose_GIcRIGdP
	  subAction knowrob;PickingUpAnObject_CTGdupxa
      subAction knowrob;PuttingDownAnObject_uBRtvLcg
      arm pr2;pr2_right_arm
      endTime knowrob;timepoint_1526640332.733415
      nextAction knowrob;TRANSPORTING_HRgQoeEw
      objectType rdf:resource="BOWL"
      performedInProjection true
      previousAction knowrob;TRANSPORTING_KVRJlsOG
      startTime rdf:resource="&knowrob;timepoint_1526640320.561752
	  taskContext "TableSetting"
	  taskSuccess true
\end{lstlisting}
\vspace{2mm}
\captionof{figure}{Example for a second Transporting Task}
\label{fig:actionExample2}
\end{minipage}


\begin{minipage}{\textwidth}
	\scriptsize
\begin{lstlisting}[frame=single]
  Individual: ConnectedSpaceRegion_YOXaigqs
	Facts: 
	  onPhysical knowrob:iai_kitchen_sink_area_counter_top
\end{lstlisting}
	\vspace{2mm}
	\captionof{figure}{Example for Connected Space Region}
	\label{fig:connectedSpaceRegion}
\end{minipage}


\begin{minipage}{\textwidth}
	\scriptsize
\begin{lstlisting}[frame=single]
  Individual: Pose_GIcRIGdP
    Facts: 
	  quaternion 0.0 0.0 1.0 0.0
	  translation -0.7599999904632568 1.190000057220459 0.9300000071525574
\end{lstlisting}
	\vspace{2mm}
	\captionof{figure}{Example for a Pose}
	\label{fig:pose}
\end{minipage}


\section{VR \neems}
\label{sec:vr-neem}
\lstset{style=lispcode}

This section will describe how \neems can be generated within a Virtual Reality environment and how they can be utilized within \cram plans to help a robot perform everyday household activities. The use of VR allows us as humans to show the robot an action we want it to perform within a variety of different environments. This facilitates learning of  a lot of common sense knowledge, e.g. where the objects necessary to perform a specific action are commonly stored within the environment, which objects are needed for a specific action, where the human user was standing when he was grasping a certain object, how the objects were arranged on a surface relative to one another and how the human user grasped them.  

The example scenario used here is the breakfast setting scenario. This means that the robot is supposed to set up the table with a bowl, cup and a spoon in preparation of a breakfast cereal meal. \todo{@Alina: move this probably to a different section.}
% Maybe make a section describing all the advantages it brings to use VR? But maybe this is also enough

\subsection{Prerequisite}
%Everythhing needed to be able to record NEEMs. Unreal, Plugins, Kitchen, RobCog, robcog_knowrob (for quering). Refer to RobCog and some of the CRAM-VR Tutorials. Also setup kitchen with all the arrays/Links.
Before VR-\neem generation can take place, the proper VR-environment needs to be set up within the Unreal Engine, including the installation of the USemLog Plugin, which records the \neems and generates the appropriate .owl files. The plugin and a setup of a kitchen environment can be found within the RobCog project. 
The KnowRob and MongoDB installation are the same as in the section above. \todo{@Alina add links to the sections}
In order to be able to use the \neems within \cram, the data first needs to be transferred into the MongoDB and KnowRob. This can be achieved by running the \textit{vr neems to knowrob} scripts. Please refer to the README.md for execution examples.

 
To summarize: 
\begin{itemize}
	\item Unreal Engine\footnote{\url{https://www.unrealengine.com/}} Version 4.22.3
	\item RobCog\footnote{\url{https://github.com/robcog-iai/RobCoG}}
	\item vr\_neems\_to\_knowrob\footnote{\url{https://github.com/ease-crc/vr_neems_to_knowrob}} 
	\item knowrob\_robcog \todo{@Alina add link}
	\item CRAM\footnote{\url{https://github.com/cram2/cram.git}} Branch: boxy-melodic

\end{itemize}

\subsection{Recording Virtual Reality Narrative Enabled Episodic Memories}
%Run around in VR and do stuff, dump semantic map using plug in
\todo{@Alina add screenshots}
Check if all items that you want to appear in the \neems, have a tag under which they will be represented within the ontology. You can check the tag by clicking any item within the kitchen environment, and going to the \textit{Actor} section in the \textit{details} pane. Click the little arrow to expand the section, and also expand the \textit{Tags} section. There you should see something like this:

\begin{lstlisting}

SemLog;Class,IAIIslandArea;Id,tpzV6l885UGL785BwZFHYQ;

\end{lstlisting}
This string defines the class of the item and it's id. More information can be added here, depending on the item and the ontology and the level of detail desired within the classification. This is also very important when introducing new items to the Virtual Reality setup. Items which do not have these tags, will not be included in the recorded \neem data. 

Once everything is set up and all potentially new items are within the virtual reality environment, the recording can begin. First, the semantic map can be automatically generated by going to the \textit{RobCog} pane within the \textit{Unreal Engine} editor, and clicking the \textit{SemanticMap} button. 
\todo{@Alina add Screenshot}
After this, the generated semantic map should be located in the \textit{RobCog/Episodes} directory. It contains the initial state of the virtual reality environment, including the position and rotation of all the furniture objects and also their classifications. 

The next step is then to start the virtual reality simulation and perform some actions within it, e.g. setting up a breakfast table. The position of the VR-headset and controllers is being tracked the entire time, as well as the interactions of the virtual hands with the environment and objects. Picking up an object would generate a \textit{GraspingSomething}-action. Placing an object down on a table would generate a \textit{Contact}-Action between the object and the surface it has been placed upon. All these interactions can later on be queried for. 

Once all the desired actions are complete, the simulation can be stopped and an \textit{EventData\_ID} directory appears in the \textit{RobCog/Episodes} directory. It contains and \textit{EventData\_ID.owl} file, an \textit{EventData\_ID.html} file, which visualizes all the occurred events, and a \textit{RawData\_ID.json}, which contains all the information about the performed actions and events. The last file is the one that needs to be uploaded into the \textit{MongoDB}. 


\subsection{Transferring VR-NEEMs into the Knowledge Base}
%Use the scripts. Maybe put the scripts into a different repo. Maybe RobCog? -> make own knowrob_robcog fork and put it there.
Please refer to the README of these scripts \url{https://github.com/ease-crc/vr_neems_to_knowrob} in order to import the VR-\neems into KnowRob and MongoDB. More information about the import, how it generally works and why the scripts were created the way they are, please refer to: \url{http://cram-system.org/tutorials/advanced/unreal#importing_new_episode_data_into_mongodb_and_knowrob_additional_information}. 

\subsection{Using VR-NEEM Data in CRAM plans}
%Which data is used, how is it used/sampled. Add reference to Paper maybe
There is a demo within \cram which uses the data collected in VR, including a tutorial on how to run it. It can be found here: \url{http://cram-system.org/tutorials/advanced/unreal}. In this demo the robot performs a pick and place task, picking up a cup, bowl and a spoon from the sink counter, and placing them onto the kitchen island. In order to do this, \cram queries KnowRob for the following information: 

\begin{itemize}
	\item where/from which surface was the object picked up?
	\item where was the human user standing when he was picking up/placing the object?
	\item on which surface and where was the object placed? (In relation to other objects)
	\item with which hand was the object grasped?
	\item from which direction (top,left,right...) was the object grasped?
\end{itemize}

Since the virtual reality kitchen can look very different than the one the robot is acting in, all of the poses are calculated relative to the respective surfaces and each other. For example, the spoon is always placed to the right of the bowl. 

For more information on how \cram interacts with KnowRob and how json-prolog can be used within \cram, please refer to \url{http://cram-system.org/tutorials/intermediate/json_prolog}




\subsection{Future Work}
%references to future work, aka. my masters thesis and ML to showcase how else VR-NEEms could be used.


